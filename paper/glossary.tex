\newacronym[description={\emph{Domain Specific Language}, język specyficzny dla
			danej dziedziny służący uściśleniu
			komunikacji}]{DSL}{DSL}{Domain Specific
	Language}

\newacronym[description={\emph{Unified Modeling Language}, język wykorzystywany
			do modelowania systemów}]{UML}{UML}{Unified Modeling
	Language}

\newacronym[description={\emph{Computation Application
				Language}, język do opisu aplikacji
			obliczeniowych w systemie
			\BalticLSC{}}]{CAL}{CAL}{Computation Application
	Language}

\newacronym[description={\emph{Eclipse Modeling Framework}, technologia
			umożliwiająca tworzenie edytorów na bazie
			metamodeli}]{EMF}{EMF}{Eclipse
	Modeling Framework}

\newacronym[description={\emph{Meta-Object Facility}, standard umożliwiający
			wytwarzanie oprogramowania sterowanego
			modelami}]{MOF}{MOF}{Meta-Object
	Facility}

\newacronym[description={\emph{Business Process Model and
				Notation}, notacja do opisu i modelowania
			procesów biznesowych}]{BPMN}{BPMN}{Business Process
	Model
	and Notation}

\newacronym[description={\emph{
				Entity--Relationship
				Model}, sposób zapisu
			relacji między tabelami w bazie danych}]{ERM}{ERM}{
	Entity--Relationship Model} % chktex 8

\newacronym[description={\emph{Visual Studio Integration Extension},
			rozszerzenie do edytora \emph{Visual Studio}
			pozwalające na tworzenie edytorów
			na bazie metamodeli}]{VSIX}{VSIX}{Visual Studio
	Integration Extension}

\newacronym[description={\emph{Extensible Markup Language}, uniwersalny język
			znaczników przeznaczony do zapisu informacji w
			ustrukturyzowany
			sposób}]{XML}{XML}{Extensible Markup Language}

\newacronym[description={\emph{Acceleo Query Language}, język wyrażeń
			umożliwiający wyrażenie dynamicznej logiki w~modelach
			\EMF{}}]{AQL}{AQL}{Acceleo Query Language}

\newacronym[description={\emph{Application Programming Interface}, interfejs
			udostępniany przez aplikację, opisujący jak~inne
			programy mogą się z nią
			komunikować}]{API}{API}{Application Programming
	Interface}

\newacronym[description={\emph{Uniform Resource Locator}, format adresowania
			zasobów w sieci Internet}]{URL}{URL}{Uniform Resource
	Locator}

\newacronym[description={\emph{Java archive}, archiwum służące kompresji i
			przechowywaniu klas języka \Java{}}]{JAR}{JAR}{Java
	archive}

\newacronym[description={\emph{Project Object Model}, model reprezentujący
			projekt w narzędziu \Maven{}}]{POM}{POM}{Project Object
	Model}

\newacronym[description={\emph{Representational state transfer}, styl
			architektury interfejsów serwerów sieciowych skupiający
			się na zasobach}]{REST}{REST}{Representational state
	transfer}

\newacronym[description={\emph{HyperText Transfer Protocol}, protokół sieci WWW
			służący do przesyłania zawartości stron
			internetowych}]{HTTP}{HTTP}{HyperText
	Transfer Protocol}

\newacronym[description={\emph{JSON Web Token}, standard żetonów w formacie
			\emph{JSON} służący uwierzytelnieniu
			użytkownika}]{JWT}{JWT}{JSON Web Token}

\newacronym[description={\emph{Cross--Origin % chktex 8
				Resource Sharing}, mechanizm
			uniemożliwiający nieautoryzowane przesyłanie zapytań do
			innych
			stron}]{CORS}{CORS}{
	Cross--Origin % chktex 8
	Resource Sharing}

\newacronym[description={\emph{Data Transfer Object}, obiekt stworzony głównie
			do przesłania danych między dwoma
			systemami}]{DTO}{DTO}{Data Transfer Object}

\newacronym[description={\emph{Depth--First % chktex 8
				Search}, algorytm przeszukiwania
			drzewa wgłąb}]{DFS}{DFS}{
	Depth--First Search} % chktex 8

\newacronym[description={\emph{Document Object Model}, interfejs pozwalający na
			dostęp do strony internetowej jako drzewa
			elementów}]{DOM}{DOM}{Document
	Object Model}

\newacronym[description={\emph{ECMAScript Module}, standard modułów języka
			\JavaScript{}}]{ESM}{ESM}{ECMAScript Module}

\newacronym[description={\emph{HyperText Markup Language}, język znaczników do
			opisu zawartości stron
			internetowych}]{HTML}{HTML}{HyperText Markup Language}

\newacronym[description={\emph{Integrated Development Environment},
			zintegrowane
			środowisko programistyczne. Program ułatwiający
			rozwijanie oprogramowania}]{IDE}{IDE}{Integrated
	Development Environment}
