\instytut{Zakład Elektrotechniki Teoretycznej i Informatyki Stosowanej}
\kierunek{Informatyka Stosowana}
\specjalnosc{Inżynieria Oprogramowania}
\title{Mechanizm walidacji modeli dla systemu BalticLSC w środowisku Sirius
	Web}
\engtitle{Validation mechanism for BalticLSC models in the Sirius Web
	environment}
\album{276887}
\author{inż.\ Grzegorz Rozdzialik}
\promotor{dr hab.\ inż.\ Michał Śmiałek}
\date{2022}
\statementplaceanddate{Warszawa, \today{} r.}
\wydzial{Wydział Elektryczny}

% W oświadczeniu o udzieleniu licencji (w folderze gfx) należy dodać tytuł,
% skreślić odpowiednie zwroty, a potem zapisać ("wydrukować") dokument jako PDF
% i zastąpić gfx/oswiadczenie-o-udzieleniu-licencji.pdf.

% Swój podpis należy wyciąć i umieścić w plik gfx/podpis.pdf
% Można go przyciąć używając programu pdfcrop

\streszczeniepracy{
	W ramach tej pracy magisterskiej został przygotowany graficzny edytor modeli
opisujących obliczenia rozproszone w systemie \BalticLSC{} bazując na
eksperymentalnej technologii \SiriusWeb{}. Została ona oceniona i porównana ze
znanym w tej kategorii rozwiązaniem \SiriusDesktop{}.

Programy komputerowe używają modeli do reprezentacji świata rzeczywistego.
W~ten~sposób mogą odwzorować rzeczywistość i symulować jej modyfikacje.
Struktura
modeli, a~więc~występujące w nich obiekty i ich właściwości, może również być
opisana
modelem. Jest~on~na~wyższym poziomie abstrakcji niż model, który opisuje, i
nosi nazwę \emph{metamodelu}.
Na jego bazie można wygenerować edytor graficzny umożliwiający przeglądanie
i edycję modeli opisywanych przez ten metamodel. \emph{\acrlong{EMF}} jest
technologią pozwalającą na przygotowanie edytora graficznego modeli
\SiriusDesktop{},
będącego rozszerzeniem natywnej aplikacji \Eclipse{}. Nowym rozwiązaniem w tej
kategorii jest \SiriusWeb{} umożliwiający udostępnienie graficznego edytora
modeli w przeglądarce internetowej.

Przygotowane rozwiązanie pozwala stworzyć model opisujący schemat obliczeń,
które będą wykonane w systemie \BalticLSC{}. Jest to platforma do obliczeń
rozproszonych, rozwijana między innymi przez Politechnikę Warszawską,
umożliwiająca przeprowadzenie obliczeń na~dużą skalę używając potencjalnie
wielu węzłów obliczeniowych jednocześnie.
Przepływ danych przez różne dostępne w systemie moduły obliczeniowe jest
reprezentowany na diagramie.

\SiriusDesktop{} i \SiriusWeb{}, pomimo bazowania na tych samych metamodelach,
oferują różne możliwości i funkcjonalności. Są również inaczej zbudowane i mają
inne podejście do~przechowywania informacji o modelach. Niektóre elementy
edytora \SiriusDesktop{}, na~przykład możliwość przeprowadzenia walidacji
semantycznej modelu, są niedostępne w~\SiriusWeb{} i zostały zaimplementowane w
tej pracy magisterskiej. W celu sprawdzenia rozszerzalności edytora dodano do
niego elementy interfejsu graficznego pozwalające wygodnie wykorzystać
dostępne moduły obliczeniowe na diagramie. Opisano również jak można
zintegrować przygotowany edytor z aplikacją przeglądarkową \BalticLSC{}.

Technologia \SiriusWeb{} zawiera również wady. Znalezionych 20 usterek zostało
zgłoszonych autorom projektu, a wrażenia z wykorzystania tego rozwiązania
zostały opisane w dalszej części tekstu pracy magisterskiej.

}
\slowakluczowe{metamodelowanie, metamodel, EMF, Eclipse Modeling Framework,
	Sirius Desktop, Sirius Web, aplikacja przeglądarkowa, edytor graficzny,
	walidacja modeli, BalticLSC}

\thesisabstract{
	{
\selectlanguage{english}
% NOTE: Interlinia ustawiona w EE-dyplom.cls nie jest stosowana dla języka
% angielskiego. Należy ją ustawić manualnie.
\setstretch{1.2213}
In this thesis a graphical model editor for expressing distributed
computations in~the~\BalticLSC{} system is presented. The editor is based on
the experimental \SiriusWeb{} technology. It~is~assessed and
compared with the more mature \SiriusDesktop{} as part of this thesis.

Computer programs use models to represent the external world. In this way they
can imitate the reality and simulate modifying it. The objects and their
attributes, which constitute the~structure of the model, can also be described
by a separate model. It has a higher abstraction level and is called a
\emph{metamodel}.
Graphical editors can be based on~it~and~allow viewing and~editing models
described by that metamodel. \emph{\acrlong{EMF}} is~a~technology used to
create graphical model editors with \SiriusDesktop{}. It is an extension
of~the~\Eclipse{} native application. A new solution in this area is
\SiriusWeb{}. It
allows creating graphical model editors running in a web browser.

The solution presented in this thesis allows creating models which describe
the~order and~types of computations run in the \BalticLSC{} system. It is a
platform for distributed computing developed, among others, by the Warsaw
University of
Technology. It allows executing large--scale % chktex 8
computations, potentially on multiple worker nodes in parallel.
Diagrams describe how~the~data flows through various available computation
units.

\SiriusDesktop{} and \SiriusWeb{}, despite being based on the same metamodels,
differ in~the~capabilities and functionalities they offer. Their architecture
and model persistence format are~also different. Some actions available in
\SiriusDesktop{}, for example invoking semantic model validation, are not
present in \SiriusWeb{} and had to be implemented as~part of~this thesis.
To assess the editor's extensibility, a new user interface feature was added.
It~allows conveniently using the available computation modules in the
diagram. The~steps required to embed the prepared solution into the
\BalticLSC{} web application were also described in this thesis.

The \SiriusWeb{} technology has its downsides. The 20 issues found within this
thesis were reported to the project's maintainers. The technology was also
assessed in the thesis text.
}

}
\thesiskeywords{metamodeling, metamodel, EMF, Eclipse Modeling Framework,
	Sirius Desktop, Sirius Web, web application, graphical editor,
	model validation, BalticLSC}
