\instytut{Zakład Elektrotechniki Teoretycznej i Informatyki Stosowanej}
\kierunek{Informatyka Stosowana}
\specjalnosc{Inżynieria Oprogramowania}
\title{Mechanizm walidacji modeli dla systemu BalticLSC w środowisku Sirius
	Web}
\engtitle{Validation mechanism for BalticLSC models in the Sirius Web
	environment}
\album{276887}
\author{inż.\ Grzegorz Rozdzialik}
\promotor{dr hab.\ inż.\ Michał Śmiałek}
\date{2022}
\statementplaceanddate{Warszawa, \today{} r.}
\wydzial{Wydział Elektryczny}

% W oświadczeniu o udzieleniu licencji (w folderze gfx) należy dodać tytuł,
% skreślić odpowiednie zwroty, a potem zapisać ("wydrukować") dokument jako PDF
% i zastąpić gfx/oswiadczenie-o-udzieleniu-licencji.pdf.

% Swój podpis należy wyciąć i umieścić w plik gfx/podpis.pdf
% Można go przyciąć używając programu pdfcrop

\streszczeniepracy{
	W ramach tej pracy magisterskiej został przygotowany graficzny edytor modeli
opisujących obliczenia rozproszone w systemie \BalticLSC{} bazując na
eksperymentalnej technologii \SiriusWeb{}. Została ona oceniona i porównana ze
znanym w tej kategorii rozwiązaniem \SiriusDesktop{}. Do bazowego edytora
\SiriusWeb{} zostały też dodane brakujące mechanizmy walidacji edytowanych
modeli.

Dla języka \emph{\acrfull{CAL}} pozwalającego reprezentować aplikacje
obliczeniowe w systemie \BalticLSC{} został przygotowany metamodel w formacie
\Ecore{} używanym w technologii \emph{\acrfull{EMF}}. Został on następnie użyty
do~przygotowania graficznego edytora modeli za pomocą \SiriusWeb{}. Wczesna
faza rozwoju tej technologii i brak jej dokumentacji technicznej spowodował, że
wymagało to~wykorzystania wiedzy z wielu różnych źródeł, a także
metody prób i błędów, aby dostosować pakiety języka \Java{} wygenerowane za
pomocą \emph{\acrshort{EMF}} do formatu wymaganego przez \SiriusWeb{}.

Prezentowany metamodel języka \emph{\acrshort{CAL}} zawiera w sobie reguły
walidacji semantycznej wyrażone jako \emph{Semantic Validation Rule} formatu
\Ecore{}. Nie są jednak one sprawdzane przez \SiriusWeb{}, który wyświetla
użytkownikowi jedynie informacje diagnostyczne z walidacji składniowej modelu.
Do edytora dodano mechanizm walidacji semantycznej poprzez modyfikację kodu
serwera aplikacyjnego i uruchamianie w nim odpowiednich reguł napisanych w
języku \Java{}. Inną funkcjonalnością, o którą wzbogacono edytor, jest
przybornik pozwalający na łatwe wykorzystanie dostępnych jednostek
obliczeniowych pobieranych z platformy \BalticLSC{}.

Przygotowany edytor modeli można wykorzystać jako część dowolnej innej
aplikacji przeglądarkowej, co zademonstrowano tworząc taką przykładową
aplikację. W pracy opisany jest też proponowany plan integracji edytora z
platformą \BalticLSC{}.

Praca zawiera również ocenę technologii \SiriusWeb{} i porównanie jej z
poprzednikiem --- \SiriusDesktop{}. Zademonstrowano różnice w wyświetlaniu
elementów diagramu, możliwości modyfikacji edytora i dodawania w nim nowych
funkcjonalności, kompletności obsługi formatu \Ecore{}, łatwości w
wykorzystaniu oraz utrzymaniu. Podczas pracy znaleziono 20 usterek
w~technologii \SiriusWeb{}, które zostały zgłoszone autorom projektu.

}
\slowakluczowe{metamodelowanie, metamodel, EMF, Eclipse Modeling Framework,
	Sirius Desktop, Sirius Web, aplikacja przeglądarkowa, edytor graficzny,
	walidacja modeli, BalticLSC}

\thesisabstract{
	{
\selectlanguage{english}
% NOTE: Interlinia ustawiona w EE-dyplom.cls nie jest stosowana dla języka
% angielskiego. Należy ją ustawić manualnie.
\setstretch{1.2213}
In this thesis a graphical model editor for expressing distributed
computations in~the~\BalticLSC{} system is presented. The editor is based on
the experimental \SiriusWeb{} technology. It~is~assessed and
compared with the more mature \SiriusDesktop{} as part of this thesis.

A metamodel in the \Ecore{} format used in the \emph{\acrfull{EMF}}
technology was prepared for the \emph{\acrfull{CAL}}. It allows
representing computation applications in the \BalticLSC{} system. It was
used to prepare a~graphical model editor using the \SiriusWeb{} technology. The
early stage of
this project combined with the lack of its technical documentation meant
that attempting to~use~it~proved to be a challenge and required combining
knowledge from multiple sources and the trial and~error approach. That is
because of the steps required to adjust the
\Java{} packages generated by \emph{\acrshort{EMF}} to the format that
\SiriusWeb{} expects, which were not clearly described.

The presented \emph{\acrshort{CAL}} metamodel contains \emph{Semantic
	Validation Rules} described in the \Ecore{} format, which help detect
invalid
models. However, they are not used in \SiriusWeb{}. It only displays diagnostic
information from the syntactic model validation. A mechanism for semantic model
validation was added to the editor by modifying the application server's code.
It relies on invoking rules represented as classes in the \Java{} programming
language. Another major functionality added to the editor is a toolbox which
allows convenient use of the available computation units fetched from the
\BalticLSC{} platform.

The presented model editor can be used as a part of another web application,
which was~demonstrated by preparing such an example application. The thesis
also proposes a~plan of~integrating the editor with the \BalticLSC{}
platform.

The thesis includes an assessment of the \SiriusWeb{} technology and a
comparison with its~predecessor --- \SiriusDesktop{}. Demonstrated are the
differences in how elements are~displayed, the ease of using, modifying,
maintaining, and enhancing the editor, as well as~the~\Ecore{} format support.
20 issues of \SiriusWeb{} were found in the process of preparing this
thesis. All of them were reported to the project's maintainers.
}

}
\thesiskeywords{metamodeling, metamodel, EMF, Eclipse Modeling Framework,
	Sirius Desktop, Sirius Web, web application, graphical editor,
	model validation, BalticLSC}
