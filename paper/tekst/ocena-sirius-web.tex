\chapter{Ocena Sirius Web}

W tej pracy magisterskiej wykorzystano rozwiązanie eksperymentalne i ciągle
rozwijane rozwiązanie \emph{Sirius Web}.
Mimo wykorzystywania tych samych metamodeli \gls{EMF} co dojrzała technologia
\emph{Sirius Desktop} występują pewne różnice. Zmiana platformy aplikacji z
natywnej na przeglądarkową ma swoje zalety. W tym rozdziale technologia
\emph{Sirius Web} zostanie oceniona i porównana z jej poprzednikiem
\emph{Sirius Desktop}, zarówno z perspektywy użytkownika edytora modeli, jak i
programisty tworzącego lub modyfikującego ten edytor.

\section{Interfejs użytkownika}

Interfejs użytkownika \emph{Sirius Web} przede wszystkim dostępny jest on w
przeglądarce, która jest domyślnie zainstalowana
na większości komputerów konsumenckich. Nie jest wymagana instalacja dodatkowej
aplikacji natywnej jaką jest \emph{Sirius Desktop}, a później pobieranie
odpowiedniej definicji metamodelu. Osoba chcąca edytować diagram wystarczy że
uruchomi stronę internetową i model zostanie jej wyświetlony lub będzie mogła
stworzyć nowy model. Taka zmiana
znacznie ułatwia wdrożenie nowych osób do systemu i umożliwienie im edytowanie
lub nawet samo oglądanie diagramów. Chcąc podzielić się z inną osobą modelem
wystarczy, że wysłany zostanie odpowiedni link, co jest dużo prostszą metodą
redukującą bariery związane z przesyłaniem plików, które wcześniej było
wymagane podczas korzystania z \emph{Sirius Desktop}.

Sam interfejs użytkownika ma mniejsze możliwości dostosowywania przez
użytkownika. Układ
interfejsu jest sztywny, można co najwyżej zmienić szerokości lub wysokości
niektórych paneli lub je schować. Większe zmiany układu jak zmiana kolejności
elementów są możliwe jedynie poprzez zmianę kodu źródłowego aplikacji
przeglądarkowej. Użytkownik nie ma na nie wpływu. Można natomiast na tyle
zmodyfikować kod interfejsu użytkownika, aby zrobić wszystko to, co jest
dostępne podczas dostosowywania interfejsu w \emph{Sirius Desktop}.

Domyślny układ edytora diagramów zdaje się być bardziej intuicyjny od tego w
\emph{Sirius Desktop}. Automatycznie rozwinięte są wszystkie potrzebne sekcje
--- drzewo obiektów modelu, informacje diagnostyczne, a także szczegóły
aktualnie zaznaczonego elementu modelu. W \emph{Sirius Desktop} użytkownik
samemu musiał dostosować ten wygląd, co sprawia, że nowi użytkownicy mogli czuć
się bardziej zagubieni.

Zmianą na lepsze jest też zwiększona częstotliwość uruchamiania walidacji
modelu. W \emph{Sirius Web} informacje diagnostyczne pojawiają się po każdej
zmianie struktury modelu, podczas gdy w \emph{Sirius Desktop} należało
manualnie wywołać walidację modelu, aby zobaczyć zaktualizowane informacje.
Takie rozwiązanie pozwala użytkownikowi dużo szybciej dowiedzieć się o
problemach w modelu i je naprawić. Nie musi on także powtarzać w kółko tej
samej czynności walidacji.

Informacje diagnostyczne wyświetlane domyślnie w \emph{Sirius Web} to jedynie
błędy z walidacji składniowej modelu. Nie są uruchamiane reguły walidacji
semantycznej (\emph{Semantic Validation Rule}) zdefiniowane w metamodelu, co
zostało omówione w sekcji~\ref{sec:reguly-walidacyjne-metamodel}. Informacja o
tym problemie została zgłoszona w repozytorium projektu\footnote{
	\url{https://github.com/eclipse-sirius/sirius-components/issues/816}}.
Aby móc wykonywać walidację semantyczną modeli należy to rozwiązanie
zaimplementować samemu, co zostało zademonstrowane w
sekcji~\ref{sec:walidacja-semantyczna-modelu}.
Oznacza to, że dodanie walidacji semantycznej do modelu jest znacznie
trudniejsze niż w \emph{Sirius Desktop}, ponieważ wymaga modyfikacji kodu
serwera aplikacyjnego. Ponadto, domyślnie nie ma przygotowanej architektury na
dodawanie reguł walidacji semantycznej, co dodatkowo utrudnia ich wdrożenie.

Informacje diagnostyczne pojawiające pojawiające się w \emph{Sirius Web} nie są
jawnie związane z konkretnymi elementami diagramu. Na diagramie nie widać
oznaczeń dla elementów, dla których występują błędy lub ostrzeżenia. Jedyna
informacja pozwalająca powiązać informację diagnostyczną z konkretnym elementem
diagramu to nazwa tego elementu występująca w treści wiadomości diagnostycznej.
Jest to widoczne na rysunku~\ref{rys:validation-comparison-sirius-web}. Lepsze
rozwiązanie tego problemu występuje w
\emph{Sirius Desktop}, gdzie informacje diagnostyczne są ściśle powiązane z
konkretnym elementem diagramu, obok którego wyświetlona jest ikona błędu.
Wskazanie tej ikony powoduje wyświetlenie treści wiadomości diagnostycznej.
Jest to zaprezentowane na
rysunku~\ref{rys:validation-comparison-sirius-desktop}.
Takie rozwiązanie pozwala użytkownikowi szybciej zlokalizować niepoprawne
elementy diagramu i zrozumieć co jest w nim błędnego. W \emph{Sirius Web}
użytkownik musi poświęcić więcej uwagi i wysiłku w celu zlokalizowania
niepoprawnych elementów.

% \begin{noindent}
\begin{figure}
	\centering
	\begin{subfigure}{.49\textwidth}
		\centering
		\includegraphics[width=.99\linewidth]{./images/sirius-web-semantic-validation-direction-and-no-loops-rules.png}
		\caption{Brak oznaczeń błędów na diagramie w \emph{Sirius
      Web}}\label{rys:validation-comparison-sirius-web}
	\end{subfigure}
  \begin{subfigure}{.49\textwidth}
		\centering
		\includegraphics[width=.99\linewidth]{./images/sirius-desktop-example-semantic-validation-rule-failure.png}
		\caption{Oznaczenia błędów na diagramie w \emph{Sirius
      Desktop}}\label{rys:validation-comparison-sirius-desktop}
	\end{subfigure}

	\caption{Różnica w prezentacji informacji diagnostycznych}
\end{figure}
% \end{noindent}

Innymi elementami zdefiniowanymi w metamodelu \gls{EMF}, które nie są w pełni
wspierane w \emph{Sirius Web} są narzędzia edycji modelu. Podstawowe narzędzie
tworzenia nowego węzła w modelu działa poprawnie.
Pewne braki występują w większości pozostałych typów narzędzi. Narzędzie do
tworzenia krawędzi nie zwraca uwagi na zdefiniowane warunki określające możliwe
początki i końce połączenia (\emph{Connection Start Precondition} i
\emph{Connection Complete Precondition}). Użytkownik może stworzyć krawędź
między dwoma dowolnymi obiektami konkretnego typu i nie otrzymuje od edytora
wskazówek bazujących na semantyce modelu. Usterka została zgłoszona w
repozytorium projektu\footnote{
	\url{https://github.com/eclipse-sirius/sirius-components/issues/779}}.

Innym narzędziem związanym z krawędziami jest możliwość zmiany elementu
początkowego lub docelowego krawędzi w sposób wizualny poprzez zaznaczenie i
przesunięcie jednego z końców krawędzi. W przypadku \emph{Sirius Web} taka
operacja jest niemożliwa, ponieważ krawędzie nie mają uchwytów na swoich
końcach. Takie uchwyty są dostępne w \emph{Sirius Desktop}, co zostało pokazane
na rysunku~\ref{rys:sirius-desktop-reconnect-edge}. W wersji
przeglądarkowej tego edytora użytkownik może wybrać inny początek lub koniec
z listy elementów modelu, co jest rozwiązaniem podatnym na błędy i
wymagającym wysiłku, albo usunąć istniejącą krawędź i stworzyć nową.
Usterka została zgłoszna w repozytorium projektu\footnote{
	\url{https://github.com/eclipse-sirius/sirius-components/issues/780}
}.

% \begin{noindent}
\begin{figure}[!hb]
  \centering

  \includegraphics[width=0.5\linewidth]{./images/sirius-desktop-reconnect-edge.png}
  \caption{Uchwyty do zmiany końców krawędzi w \emph{Sirius
    Desktop}}\label{rys:sirius-desktop-reconnect-edge}
\end{figure}
% \end{noindent}

Kolejnym narzędziem metamodelu, które nie jest wspierane przez \emph{Sirius
	Web} są okna dialogowe pozwalające na wytworzenie bardziej
skomplikowanych
schematów edycji modelu czy zautomatyzowanie niektórych czynności. Wyświetlenie
własnego okna dialogowego pozwalałoby na uproszczenie tworzenia nowego
wywołania modułu obliczeniowego w modelu, ponieważ użytkownik mógłby w oknie
dialogowym wskazać moduł obliczeniowy do wywołania. Takie narzędzie
przygotowano w \emph{Sirius Desktop}, co widać na
rysunku~\ref{rys:sirius-desktop-create-unit-call-dialog}. W \emph{Sirius
	Web} przy próbie wywołania tego narzędzia pojawia się błąd w konsoli
serwera
aplikacyjnego informujący, że to narzędzie nie jest zaimplementowane. Usterka
została zgłoszona w repozytorium projektu\footnote{
	\url{https://github.com/eclipse-sirius/sirius-components/issues/815}
}.

% \begin{noindent}
\begin{figure}[!hb]
  \centering

  \includegraphics[width=0.95\linewidth]{./images/sirius-desktop-create-unit-call-dialog.png}
  \caption{Okno dialogowe w \emph{Sirius Desktop} ułatwiające tworzenie nowego
    obiektu}\label{rys:sirius-desktop-create-unit-call-dialog}
\end{figure}
% \end{noindent}

W \emph{Sirius Desktop} przybornik z narzędziami metamodelu jest wyświetlony
w osobnym oknie, co widać po prawej stronie
rysunku~\ref{rys:sirius-desktop-model-editor-tools-right}. Zawiera on w każdym
momencie wszystkie dostępne narzędzia.
Użytkownik musi nauczyć się które narzędzia są dostępne dla których obiektów
modelu i kiedy mogą zostać wykorzystane. Przykładowo, po wybraniu narzędzia
\emph{Data Flow} służącego tworzeniu połączeń między portami użytkownik może
wskazać jedynie jeden z portów. Brakuje wskazówek co należy dalej zrobić.
\emph{Sirius Web} inaczej reprezentuje narzędzia. Są one umieszczone w
przyborniku pokazywanym po kliknięciu myszą na element diagramu. Wyświetlone są
jedynie narzędzia związane z wybranym elementem, co upraszcza wybór, ponieważ
użytkownik nie musi wiedzieć które narzędzia mają sens dla wybranego elementu
--- \emph{Sirius Web} automatycznie ogranicza wybór. Jest to rozwiązanie lepsze
niż w \emph{Sirius Desktop} pod względem intuicyjności interfejsu.

% \begin{noindent}
\begin{figure}[!hb]
  \centering

  \includegraphics[width=0.95\linewidth]{./images/sirius-desktop-model-editor.png}
  \caption{Interfejs \emph{Sirius Desktop} z narzędziami po prawej
    stronie}\label{rys:sirius-desktop-model-editor-tools-right}
\end{figure}
% \end{noindent}

W \emph{Sirius Web} występują 2 problemy związane z przybornikiem narzędzi. Po
pierwsze niektóre narzędzia są wyświetlane podwójnie. Można to zobaczyć na
rysunku~\ref{rys:sirius-web-duplicate-tools}. Pokazane są dwie ikony ze
strzałką służące tworzeniu połączenia wychodzącego z portu \emph{output}, a
także dwie ikony związane z usunięciem elementu (\emph{DeleteComputedDataPin}
oraz ikona kosza). Taka duplikacja może zaskoczyć użytkownika i sprawić, że nie
będzie on wiedział który przycisk należy wybrać, aby zrealizować oczekiwany cel
oraz jakie są różnice między tymi przyciskami. W rzeczywistości oba przyciski w
obu przypadkach mają taki sam skutek, więc duplikacja jest niepotrzebna.

% \begin{noindent}
\begin{figure}[!ht]
  \centering

  \includegraphics[width=0.95\linewidth]{./images/sirius-web-duplicate-tools.png}
  \caption{Zduplikowane narzędzia w \emph{Sirius
    Web}}\label{rys:sirius-web-duplicate-tools}
\end{figure}
% \end{noindent}

Kontynuując temat przybornika w \emph{Sirius Web}, z
rysunku~\ref{rys:sirius-web-duplicate-tools} wynika, że dla portu wywołania
modułu obliczeniowego pokazane są dwa narzędzia dotyczące jego usunięcia. W
metamodelu rzeczywiście jest takie narzędzie, ale jest ono tam wyłącznie po to,
aby zabronić usuwania tych elementów z diagramu, ponieważ są one zarządzane
automatycznie i ich usunięcie natychmiast sprawiłoby, że diagram stałby
się niepoprawny semantycznie (wywołanie modułu obliczeniowego miałoby mniej
portów niż powiązany z nim moduł obliczeniowy). W \emph{Sirius Desktop}
usunięcie tego portu z diagramu jest niemożliwe, co widać na
rysunku~\ref{rys:sirius-desktop-blocked-deleting-pins}.
\emph{Sirius Web} wyświetla odpowiednie ikony i pozwala je wybrać, co skutkuje
wyświeleniem błędu w konsoli serwera aplikacyjnego, a po kilku sekundach
wyświetleniem niejasnej wiadomości o przekroczeniu czasu przetwarzania żądania,
przedstawionego na rysunku~\ref{rys:sirius-web-timeout-when-deleting-pins}.
Takie rozwiązanie jest mylące dla użytkownika i pogarsza wrażenia z
wykorzystania edytora.

% \begin{noindent}
\begin{figure}[!hb]
  \centering

  \includegraphics[width=0.95\linewidth]{./images/sirius-desktop-blocked-deleting-pins.png}
  \caption{Zablokowana możliwość usunięcia portu wywołania modułu
    obliczeniowego w \emph{Sirius Desktop}}\label{rys:sirius-desktop-blocked-deleting-pins}
\end{figure}
% \end{noindent}

% \begin{noindent}
\begin{figure}[!hb]
  \centering

  \includegraphics[width=0.95\linewidth]{./images/sirius-web-timeout-when-deleting-pins.png}
  \caption{Błąd podczas próby usunięcia portu wywołania modułu obliczeniowego w
    \emph{Sirius Web}}\label{rys:sirius-web-timeout-when-deleting-pins}
\end{figure}
% \end{noindent}

Niektóre elementy na diagramie wyświetlane są w inny sposób. Porty, które w
metamodelu zostały oznaczone jako węzły krawędziowe (\emph{Border Node}) w
\emph{Sirius Desktop} rzeczywiście są wyświetlane na krawędzi swojego rodzica
(węzła wywołania modułu obliczeniowego), natomiast w \emph{Sirius Web} są
wyświetlane w jego środku. Usterka jest znana i została zgłoszona\footnote{
	\url{https://github.com/eclipse-sirius/sirius-components/issues/956}
}. Różnice w wyświetlaniu węzłów krawędziowych zostały zaprezentowane na
rysunku~\ref{rys:border-node-difference}.
Przy okazji warto zauważyć, że \emph{Sirius Web} nie obsługuje wyświetlania
krawędzi węzłów krawędziowych. Te widoczne na
rysunku~\ref{rys:border-node-sirius-web} pochodzą z ikony i zostały specjalnie
dodane po odkryciu tej usterki. Została ona zgłoszna w repozytorium
projektu\footnote{
	\url{https://github.com/eclipse-sirius/sirius-components/issues/783}}.

% \begin{noindent}
\begin{figure}
	\centering
	\begin{subfigure}{.49\textwidth}
		\centering
		\includegraphics[width=.99\linewidth]{./images/border-node-sirius-desktop.png}
		\caption{Węzły krawędziowe w \emph{Sirius Desktop} są wyświetlane na
      krawędzi}
	\end{subfigure}
	\begin{subfigure}{.49\textwidth}
		\centering
		\includegraphics[width=.99\linewidth]{./images/border-node-sirius-web.png}
		\caption{Węzły krawędziowe w \emph{Sirius Web} są wyświetlane wewnątrz
      węzłą}\label{rys:border-node-sirius-web}
	\end{subfigure}

    \caption{Różnica w wyświetlaniu węzłów
      krawędziowych}\label{rys:border-node-difference}
\end{figure}
% \end{noindent}

Ikony obiektów metamodelu również są wyświetlane inaczej niż w \emph{Sirius
	Desktop}. Jeżeli rozmiar ikony przekracza 16 pikseli na 16 pikseli,
ikona
zaczyna pojawiać się pod etykietą obiektu. Porównanie zachowania obu edytorów w
tym przypadku zostało przedstawione na
rysunku~\ref{rys:sirius-multiline-labels}. Problem został zgłoszony w
repozytorium projektu\footnote{
	\url{https://github.com/eclipse-sirius/sirius-components/issues/782}
}. Jako próba jego obejścia zmniejszono ikony do rozmiaru 16 na 16 pikseli, aby
tekst był czytelny.

% \begin{noindent}
\begin{figure}
	\centering
	\begin{subfigure}{.49\textwidth}
		\centering
    \includegraphics[width=.99\linewidth]{./images/sirius-desktop-multiline-label.png}
		\caption{Dwuwierszowa etykieta obiektu w \emph{Sirius Desktop}}
	\end{subfigure}
	\begin{subfigure}{.49\textwidth}
		\centering
		\includegraphics[width=.99\linewidth]{./images/sirius-web-multiline-label.png}
		\caption{Ta sama etykieta w \emph{Sirius Web}}\label{rys:border-node-sirius-web}
    \medskip
    {\footnotesize Ponadto, ikona o rozmiarze większym niż $16\times16$ pikseli nachodzi na tekst.}
	\end{subfigure}

    \caption{Wyświetlanie dwuwierszowych
      etykiet oraz większych ikon}\label{rys:sirius-multiline-labels}
\end{figure}
% \end{noindent}

Pozostając w temacie etykiet obiektów są z nimi 2 problemy. Po pierwsze,
znaki nowej linii w etykietach są zamieniane na zwykłe odstępy. Widać to na
rysunku~\ref{rys:sirius-multiline-labels}, gdzie w \emph{Sirius Desktop} nazwa
wywoływanego modułu
obliczeniowego znajduje się w nowej linii, natomiast w \emph{Sirius Web} cała
etykieta to jedna linia. Zaburza to estetykę diagramu. Usterka została
zgłoszona w repozytorium projektu\footnote{
	\url{https://github.com/eclipse-sirius/sirius-components/issues/781}
}. Drugim z problemów związanym z etykietą jest jej edycja w sytuacji, gdy
etykieta jest obliczana w języku \gls{AQL} na podstawie kilku właściwości
obiektu. \emph{Sirius Desktop} nie pozwala na bezpośrednią edycję całej
etykiety w tym przypadku, a \emph{Sirius Web} wyświetla pole do zmiany tekstu
tylko po to, aby go później zignorować --- po zatwierdzeniu zmiany etykieta
wraca do swojej początkowej wartości, a w konsoli serwera aplikacyjnego
wyświetlany jest błąd. Usterka została zgłoszona w repozytorium
projektu\footnote{
	\url{https://github.com/eclipse-sirius/sirius-components/issues/784}
}.

Warunkowe zmiany stylu elementów modelu (\emph{Style Customizations}) działają
poprawnie. Dzięki nim ikony portów w \emph{Sirius Web} zmieniają się na
podstawie swoich parametrów (krotności danych i tokenów). Drugi z stylów
warunkówych w metamodelu dotyczący zmiany koloru krawędzi portu ze względu na
jego typ (wejściowy lub wyjściowy) nie może zostać zweryfikowany, ponieważ
krawędzie te w ogóle nie są wyświetlane.

\emph{Sirius Web} nie wprowadza niektórych ograniczeń, które są przestrzegane w
\emph{Sirius Desktop}. Jednym z nich jest zablokowanie możliwości zmiany
rozmiaru węzłów. W metamodelu zabroniono zmiany rozmiaru portów i takiego
zachowania można się spodziewać w aplikacji natywnej, natomiast w \emph{Sirius
	Web} zmiana ich rozmiaru jest możliwa dowolnie, co zostało
przedstawione na
rysunku~\ref{rys:change-node-size}. Problem został zgłoszony w repozytorium
projektu\footnote{
	\url{https://github.com/eclipse-sirius/sirius-components/issues/785}}.

% \begin{noindent}
\begin{figure}[!hb]
  \centering

  \includegraphics[width=0.6\linewidth]{./images/change-node-size.png}
  \caption{Zmieniony rozmiar węzłów w Sirius Web}\label{rys:change-node-size}
\end{figure}
% \end{noindent}

Interfejs użytkownika \emph{Sirius Web} zazwyczaj reguje na zmiany natychmiast.
Animacje używane w edytorze diagramów występujące podczas dodawania lub
usuwania z niego elementów sprawiają, że aplikacja jest przyjemniejsza w użyciu
i sprawia wrażenie bardziej dopracowanej, a jednocześnie pomaga zrozumieć co
się dzieje w modelu. Pomaga to szczególnie, gdy zmiany są wykonywane przez
inną osobę edytującą ten diagram w czasie rzeczywistym.
Problem z wydajnością występuje podczas przesuwania elementów w \emph{Sirius
	Web}. Interfejs przesuwa elementy z opóźnieniem, a po upuszczeniu
elementów
często ignoruje przyciśnięcia klawiszy myszy przez około sekundę. Jest to
wyjątek pośród ogólnie płynnego interfejsu, co dodatkowo uwidacznia ten
problem.

Udostępnienie aplikacji do edycji diagramów w przeglądarce powoduje, że na
jej wydajność wpływa szybkość łącza Internet użytkownika. Architektura
rozwiązania \emph{Sirius Web} powoduje, że aplikacja często musi komunikować
się z serwerem aplikacyjnym. Dzieje się tak z uwagi na fakt, że informacje o
modelu przechowywane są w bazie danych a metamodel \gls{EMF} jest
interpretowany przez serwer. Do przeglądarki komunikowane są jedynie informacje
o modelu, a nie metamodel. Dla przykładu, wywołanie narzędzia tworzącego nowe
wywołanie modułu obliczeniowego wymaga przesłania zapytania typu
\emph{mutation} do serwera aplikacyjnego, otrzymania odpowiedzi potwierdzającej
tą zmianę, po czym asynchronicznie serwer wysyła jeszcze 3 wiadomości
za pomocą połączenia \emph{GraphQL} \emph{subscription}:

\begin{itemize}
	\item \texttt{diagramEvent} z nową zawartością diagramu,
	\item \texttt{validationEvent} z nowym zestawem informacji
	      diagnostycznych,
	\item \texttt{treeEvent} z nową zawartością drzewa modelu.
\end{itemize}

Te 3 wiadomości zawierają informacje o \textbf{całym} modelu. W przeciwieństwie
do podejścia przyrostowego, w którym przesyłane byłyby jedynie informacje o
zmianach względem poprzedniego stanu modelu, podejście z przesyłaniem
wszystkich informacji za każdym razem powoduje, że przez sieć przesyłanych jest
znacznie więcej danych. Ponadto, aplikacja przeglądarkowa musi je
zinterpretować i odpowiednio zaaplikować w interfejsie, co również wymaga
większej ilości obliczeń. Znaczenie tego problemu jest proporcjonalne do
wielkości modelu --- im większy model, tym więcej informacji jest przesyłanych
przez sieć przy każdej jego modyfikacji.

Aplikacja działała płynnie podczas korzystania z niej w sytuacji, w której
serwer aplikacyjny był uruchomiony na komputerze, na którym wykonywano test.
Opóźnienia wynikające z narzutu na przesyłanie danych przez sieć były
zauważalne, gdy została sztucznie ograniczona szybkość połączenia internetowego
za pomocą narzędzi programisty w przeglądace \emph{Google Chrome}.
Używanie ustawienia \emph{Fast 3G}, które ogranicza szybkość pobierania danych
do 1.5~Mb/s i opóźnieniu
500~ms~\cite{network-throttling-profiles-stackoverflow} sprawiło, że aplikacja
przestała być używalna z uwagi na długi czas pobierania danych o modelu.
Szczegółowe czasy oczekiwania na wyświetlenie elementów interfejsu z włączonym
sztucznym opóźnieniem zostały zaprezentowane w
tabeli~\ref{tab:sirius-web-ui-delay-throttled}.

\begin{table}[!b]
	\centering
	\begin{tabular}{p{5cm}c}
		\toprule
		Element interfejsu       & Czas oczekiwania (sekundy) \\
		\midrule
		Drzewo modelu            & 92                         \\
		Informacje diagnostyczne & 92                         \\
		Diagram                  & 154                        \\
		\bottomrule
	\end{tabular}
	\caption{Czas oczekiwania na wyświetlenie elementów interfejsu
		\emph{Sirius
			Web} z włączonym sztucznym opóźnieniem \emph{Fast
			3G}}\label{tab:sirius-web-ui-delay-throttled}
\end{table}

Opóźnienie sieciowe nie ma wpływu na modyfikację pozycji elementów. Użytkownik
może płynnie przesuwać węzły na diagramie. Wszelkie inne operacje wymagają
informacji z serwera aplikacyjnego. Oznacza to, że korzystanie z edytora
\emph{Sirius Web} wymaga szybkiego i stabilnego połączenia sieciowego. Nie jest
to wymagane w przypadku \emph{Sirius Desktop}, w którym wydajność zależy
wyłącznie od szybkości komputera używanego do edycji modeli.

Możliwym rozwiązaniem problemu przesyłania zbyt dużej ilości danych przez sieć
w \emph{Sirius Web} byłoby zmiana zawartości przesyłanych wiadomości na
podejście przyrostowe. Komunikaty zawierałyby jedynie informacje o zmianach,
które zazwyczaj można wyrazić zwięźlej niż przesyłając cała zawartość
zaktualizowanego modelu.

\section{Błędy techniczne}

% TODO:
% * Brak możliwości dodania uwierzytelniania do GraphQL Subscription
% * Omówić pozostałe błędy techniczne z Notion
% * Powiedzieć, że niektóre błedy zostały naprawione na GitHubie

% https://github.com/eclipse-sirius/sirius-components/issues/created_by/Gelio
% https://www.notion.so/gregorr/R-nice-wzgl-dem-Sirius-Desktop-21b79f2342c647c5b35a147a78bfe3f3

\section{Użycie własnego metamodelu}

% TODO:
% * Brak dokumentacji jak użyć własny model
% * Opisać moje doświadczenia

\section{Dodawanie funkcjonalności do edytora}

% TODO
% * Opisać moje doświadczenia ogólnie
% * Trudności w modyfikacji, bo nie wszystkie komponenty są wyeksportowane.
% Konieczność kopiowania kodu
% * Trudności w aktualizowaniu do nowych wersji, bo kod się mocno zmienia.
% Szczególnie trudno jak się skopiowało coś z sirius-components
% * Brak dokumentacji jak można rozszerzać edytor oraz jak on działa. Wymaga
% reverse engineering
