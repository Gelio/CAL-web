\chapter{Podsumowanie}

W tym rozdziale zostaną opisane wyniki pracy magisterskiej oraz wnioski
wyciągnięte podczas jej realizacji, a także możliwości dalszy rozwój
przygotowanego rozwiązania.

\section{Wyniki pracy}

W ramach tej pracy magisterskiej technologia \SiriusWeb{} została
wykorzystana do stworzenia edytora diagramów opisujących aplikacje obliczeniowe
systemu \BalticLSC{}. Wykorzystuje on~przygotowany specjalnie w tym celu
metamodel języka \CAL{} w formacie \Ecore{}. Do~edytora zostały dodane
narzędzia i
funkcjonalności pomagające w efektywniejszym i szybszym tworzeniu modeli.
Ocenione zostały także możliwości edytora i porównane z możliwościami
konkurencyjnej poprzedniej wersji edytora bazującego na technologii \EMF{},
czyli \SiriusDesktop{}.

Edytor użyty jest domyślnie jako część interfejsu przykładowej aplikacji
wykorzystującej bibliotekę \React{}.
Jako jedna z części tej pracy udostępniono interfejs programistyczny
pozwalający na wyświetlenie edytora w dowolnej aplikacji przeglądarkowej
wykorzystując wyłącznie funkcje w języku \JavaScript{}, bez wymogu używania
konkretnej biblioteki.
Pozwala to~na~wykorzystanie go jako
alternatywny edytor diagramów platformy \BalticLSC{}, która wykorzystuje
bibliotekę \emph{Vue.js}. Zostało to
zademonstrowane w sekcji~\ref{sec:przykladowa-aplikacja-wykorzystujaca-edytor}
poprzez przygotowanie aplikacji przeglądarkowej wyświetlającej
edytor diagramów jako część swojego interfejsu.

Edytor domyślnie pozwala na wyświetlenie modeli w formie diagramów, zgodnie z
definicją jego reprezentacji z metamodelu w formacie \Ecore{}, oraz na
podstawowe możliwości
edycji tych modeli. Zachowana jest funkcjonalność interpretacji języka
\emphgls{AQL} do wyrażenia bardziej zaawansowanej logiki czy wyrażeń
opisujących
właściwości metamodelu. Jednak nie wszystkie możliwości formatu \Ecore{}
wspierane przez \SiriusDesktop{} są obsługiwane przez \SiriusWeb{}.
Niektóre elementy są również inaczej wyświetlane co sprawia, że metamodel
należy szczególnie dostosować pod \SiriusWeb{}.

Różnice między \SiriusWeb{} i \SiriusDesktop{}, a także błędy w
funkcjonowaniu oraz~brakujące funkcjonalności edytora sprawiły, że autorom
projektu zostało zgłoszonych 20~usterek w~repozytorium projektu
\texttt{sirius-components} na platformie \GitHub{}. Dwie z nich zostały
rozwiązane podczas pisania pracy magisterskiej. Część problemów posiadała
alternatywne rozwiązania, które pomimo swoich wad pozwalały uzyskać zamierzony
efekt. W innych przypadkach brakujące funkcjonalności należało zaimplementować
samemu lub zaakceptować istnienie problemu.

Funkcjonalnościami dodanymi do edytora w ramach pracy magisterskiej są
mechanizm walidacji semantycznej modeli języka \CAL{} oraz implementacja
przybornika wyświetlającego listę dostępnych jednostek obliczeniowych z
platformy
\BalticLSC{}. Modyfikacja kodu serwera aplikacyjnego zazwyczaj była
łatwiejsza od modyfikacji kodu aplikacji przeglądarkowej, ponieważ dzięki
mechanizmowi wstrzykiwania zależności z platformy \emph{Java Spring}
często wystarczyło jedynie dodać nowe klasy i ewentualnie zmienić kilka linii w
już istniejących.

Zmiana kodu interfejsu użytkownika aplikacji przeglądarkowej była pracochłonna
i wiązała się ze skopiowaniem kodu źródłowego tych komponentów, które mają być
modyfikowane. \SiriusWeb{} nie udostępnia prostych metod na dodanie
własnych elementów interfejsu do istniejących komponentów, więc kopiowanie i
edycja ich kodu źródłowego jest jedynym rozwiązaniem. Dodaje to natomiast pracy
przy późniejszych aktualizacjach do nowszych wersji, ponieważ należy upewnić
się
czy kod skopiowanych komponentów został zmieniony. Jeżeli tak, należy odtworzyć
te zmiany, ponieważ w przeciwnym wypadku aplikacja może działać niepoprawnie.

Wykorzystanie, a później modyfikacja edytora \SiriusWeb{} była utrudniona
przez brak dokumentacji, niewielką liczbę publicznie dostępnych przykładów
wykorzystania rozwiązania oraz małą liczbę użytkowników zadających pytania,
których odpowiedzi byłyby widoczne publicznie. Te wszystkie aspekty wynikają z
krótkiego czasu życia projektu, ponieważ został on rozpoczęty w 2018 roku, a
dopiero w 2020 roku jego kod źródłowy został udostępniony na platformie
\GitHub{}. Praca nad edytorem wymagała poświęcenia sporej ilości czasu na
zrozumienie jak \SiriusWeb{} działa poprzez czytanie kodu źródłowego
przykładowej aplikacji z repozytorium \texttt{sirius-web}, jak i poszczególnych
klas biblioteki z repozytorium \texttt{sirius-components}. Gdy znalezienie
odpowiedzi na nurtujące pytania zajmowało zbyt dużo czasu, zadawano pytanie
autorom projektu za pomocą \emph{GitHub Issues}.

Umożliwienie edycji modeli w przeglądarce ma sporo zalet w porównaniu do
edytorów funkcjonujących jako aplikacje natywne, takie jak \SiriusDesktop{}.
Nie wymaga ona procesu instalacji, więc można szybciej
udostępnić
model innym użytkownikom. Łatwiej jest przeprowadzać aktualizację metamodelu do
nowszej wersji. Zmiany w modelach są automatycznie propagowane do wszystkich
użytkowników w czasie rzeczywistym. Sam interfejs został też~odświeżony i
wygląda bardziej estetycznie w porównaniu do \SiriusDesktop{}.

Wykorzystanie sieci Internet do obsługi edytora ma też swoje wady. Niektóre z
nich są~szczególnie problematyczne i widoczne z uwagi na przyjęty schemat
komunikacji między aplikacją przeglądarkową a serwerem aplikacyjnym. Na wolnych
lub niestabilnych łączach aplikacja edytor może działać ze sporym opóźnieniem i
powodować frustrację podczas jego obsługi, bo każda zmiana modelu musi zostać
wprowadzona przez serwer aplikacyjny. Problem pogarsza fakt, że przy każdej
zmianie z serwera przesyłane są wszystkie informacje o modelu zamiast jedynie
informacji o zmienionych fragmentach.

Centralizacja przechowywania modeli w obrębie serwera aplikacyjnego często
oznacza, że~pożądaną funkcjonalnością jest kontrola dostępu użytkowników do
modeli. W organizacjach użytkownicy powinni mieć ograniczony dostęp do zasobów
i być uwierzytelnieni. \SiriusWeb{} domyślnie nie narzuca żadnych
ograniczeń dostępu do modeli, jednak większość operacji można zabezpieczyć za
pomocą rozwiązania \emph{Spring Security}. Jedynym kanałem, który nie~może
zostać
zabezpieczony, są subskrypcje \GraphQL{}, które wymagają specjalnej metody
uwierzytelnienia. Za ich pomocą dowolna osoba może otrzymać informacje o
zawartości dowolnego modelu. Jest to poważna wada, ponieważ nie można wtedy
zapewnić kontroli dostępu, co jest podstawowym wymaganiem w aplikacjach
wykorzystywanych w organizacjach. Prawdopodobnie skreśla to \SiriusWeb{}
jako możliwe rozwiązanie chyba, że~wdrożone są~dodatkowe mechanizmy
bezpieczeństwa nieumożliwiające dostęp do serwera aplikacyjnego osobom
nieuwierzytelnionym.

Projekt \SiriusWeb{} jest aktywnie rozwijany. W ciągu miesiąca
(od~15~grudnia~2021~r.~do~15~stycznia~2022~r.) zostało wydanych 9 nowych
wersji tego
oprogramowania. Niektóre z nich wymagały jedynie pobrania nowych plików
źródłowych, a inne z nich wprowadzały zmiany łamiące kompatybilność wsteczną i
wymagające niekiedy znacznych zmian w kodzie źródłowym. Zmiany te były najpierw
wprowadzane przez zespół pracujący nad tą technologią w~repozytorium
\texttt{sirius-web}. Można było te zmiany wykorzystać podczas aktualizacji
własnej kopii edytora. Było to natomiast tym trudniejsze, im bardziej edytor
został przez nas zmodyfikowany, ponieważ wtedy kod bardziej się różnił i
narzędzia umożliwiające automatyczne wprowadzanie zmian (pochodzące z systemu
kontroli wersji \emph{git}) nie wiedziały jak je wprowadzić.

Podsumowując, wszystkie zamierzone cele pracy magisterskiej zostały osiągnięte.

\section{Wnioski}

\SiriusWeb{} jest technologią eksperymentalną, nieposiadającą jeszcze
dokumentacji ani wielu przykładów zastosowania. Brakuje też listy
nieobsługiwanych funkcjonalności w porównaniu do~\SiriusDesktop{}. Praca z
tego typu projektem wprowadza ryzyko, ponieważ trudniej oszacować czas
i wysiłek potrzebny na jego wykorzystanie czy wdrożenie nowych funkcjonalności.
Niektóre proste z pozoru dodatki mogą okazać się nietrywialne w wykonaniu z
uwagi na sposób działania rozwiązania.

\SiriusWeb{} domyślnie nie wyświetla informacji diagnostycznych pochodzących z
reguł walidacji semantycznej (\emph{Semantic Validation Rule}) umieszczonych w
metamodelu formatu \Ecore{}. W interfejsie użytkownika pokazywane są jedynie
wyniki walidacji składniowej. Pełna walidacja modeli wymaga dodania własnego
mechanizmu walidacji semantycznej, na~przykład poprzez stworzenie klas w języku
\Java{} reprezentujących reguły walidacyjne, a~później uruchamianie ich~podczas
walidacji w \SiriusWeb{}. Dodanie obsługi reguł walidacji semantycznej
z~metamodelu w formacie \Ecore{} ułatwiłoby przygotowywanie kompletnego
i~bardziej przydatnego graficznego edytora modeli.

Ryzyko, które należy brać pod uwagę szczególnie wykorzystując nowe
technologie jest możliwość znalezienia usterek, które będą wymagały czasu na
ich naprawienie. Większość z problemów zgłoszonych w projekcie \SiriusWeb{}
nie zostało naprawionych pomimo, że~od~ich~zgłoszenia minęły w niektórych
przypadkach nawet 3 miesiące. Planując projekt warto uwzględnić na jego samym
początku czas na sprawdzenie możliwości wykorzystywanych narzędzi. W przypadku
niepewności należy przygotować przykład ich zastosowania, który pomógłby w
znalezieniu problemów lub brakujących funkcjonalności na wczesnym etapie
projektu. W~ten~sposób można wcześniej zmienić harmonogram projektu i podjąć
działania minimalizujące negatywne efekty, jak na przykład przeznaczenie
większej ilości czasu na~naprawę usterek albo całkowita rezygnacja z niektórych
funkcjonalności.

Rozpoczynając pracę z nowymi technologiami warto najpierw poznać podstawy
ich~działania, a dopiero później próbować je w znaczącym stopniu zmodyfikować.
Warto także spróbować wykorzystać je osobno zanim spróbuje się je
zintegrować ze sobą. Tak było z~\emph{Java Spring}, \Maven{} oraz
\EMF{}, z którymi autor pracy nie miał wcześniej doświadczenia. Próby
modyfikacji bez znajomości zasad działania tych technologii prowadziły do
błędów, często trudnych do zrozumienia, oraz frustracji. Spędzenie kilku godzin
na poznawaniu tych technologii i~wykorzystaniu ich w izolacji przed
modyfikowaniem \SiriusWeb{} używającego ich~wszystkich sprawiło, że
szybciej wykonano postępy.

Mając do czynienia z dwoma podobnymi technologiami korzystającymi z tego samego
źródła trzeba mieć na uwadze możliwe niekompatybilności. Aby je zweryfikować
należy wykorzystać je~zaczynając od najprostszych funkcjonalności zanim
przejdzie się do tych
bardziej skomplikowanych. Pozwoli to zaoszczędzić czas spędzony na
dostosowywaniu jednego z~rozwiązań podczas gdy drugie nie będzie w stanie
obsłużyć tych samych funkcjonalności. Tak było w przypadku wsparcia różnych
składników formatu \Ecore{} przez \SiriusDesktop{} i \SiriusWeb{}. Najpierw
zainwestowano czas w przygotowywanie kompletnego
metamodelu z
wykorzystaniem \SiriusDesktop{}, po czym okazało się, że część
funkcjonalności nie jest wspierana przez \SiriusWeb{} lub jest wspierana
częściowo i metamodel wymaga zmian. Zmarnowano w ten sposób czas
na~przygotowywanie elementów, które ostatecznie nie były potrzebne, ponieważ
celem
pracy było wykorzystanie metamodelu wyłącznie w \SiriusWeb{}.

Wykorzystanie technologii \emph{Docker}~\cite{wikipedia-docker} znacznie
ułatwia rozpoczęcie pracy z
nowymi narzędziami oraz wdrożenie projektu.
Stworzenie obrazu \emph{Docker} zawierającego serwer aplikacyjny \SiriusWeb{}
umożliwiło dużo szybsze wdrożenie aplikacji oraz jej aktualizację
do~nowej wersji. Wystarczyło zainstalować silnik \emph{Docker}, a później
pobrać i uruchomić obraz. Nie~było wymagane instalowanie środowiska \Java{}
wewnątrz maszyny wirtualnej. Nie~występowały też błędy związane z
kompatybilnością czy brakiem narzędzi zainstalowanych na jednym ze~środowisk.
Warto rozważyć technologię \emph{Docker} w projektach, aby ułatwić ich
wdrożenie.

\section{Możliwości na rozwój}

Wyniki pracy magisterskiej można rozwijać na kilka sposobów. Niektóre
możliwości rozwoju zostały omówione w tej sekcji.

\paragraph{Wdrożenie edytora w systemie \BalticLSC{}.}
Jedną z możliwości jest integracja przygotowanego edytora modeli z systemem
\BalticLSC{}. Można zastąpić dotychczasowy używany tam edytor diagramów lub
użyć \SiriusWeb{} jako alternatywnego. Taki krok umożliwiłby użytkownikom
\BalticLSC{} korzystanie z narzędzia, które pozbawione jest niektórych
usterek znanych z aktualnego rozwiązania, jak na przykład brak wsparcia
wyświetlania na diagramie jednostek obliczeniowych, które zostały usunięte z
przybornika.
Inną zaletą \SiriusWeb{} nad aktualnym edytorem jest możliwość współpracy
użytkowników nad diagramem w czasie rzeczywistym. Taka integracja wymaga zmian
w serwerach aplikacyjnych zarówno \SiriusWeb{}, jak i \BalticLSC{}.
Zostały one opisane w
sekcji~\ref{sec:pomysl-integracji-sirius-web-i-balticlsc}.

\paragraph{Dodanie kolejnych reguł walidacji semantycznej.}
Inną metodą rozwoju wyników pracy magisterskiej jest ustalenie i
zaprojektowanie większej liczby reguł walidacji semantycznej. W~ten~sposób
jeżeli edytor diagramów będzie wykorzystany w systemie \BalticLSC{}, jego
użytkownicy otrzymają więcej szczegółowych informacji diagnostycznych o błędach
lub~ostrzeżeniach w~modelu. Pomoże im to łatwiej zauważyć niepoprawne
ustawienia elementów lub~pomyłki, które zrobili.

\paragraph{Kontrybucja do projektu \SiriusWeb{}.}
Projekt \SiriusWeb{} ma otwarte źródło. Można więc pomóc autorom tego
projektu naprawić znalezione w nim usterki poprzez zaimplementowanie
brakujących funkcjonalności lub poprawę błędów. Będzie to przydatne nie tylko
dla potencjalnego wdrożenia edytora w \BalticLSC{}, ale także dla innych
osób próbujących wykorzystać \SiriusWeb{} do własnych zastosowań.
Oprócz modyfikacji kodu można także pomóc w tworzeniu dokumentacji, która jest
kluczowa do szerszego rozprzestrzenienia się projektu i~jego wykorzystania
przez inne osoby.
