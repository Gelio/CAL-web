\chapter{Podsumowanie}

W tym rozdziale zostaną opisane wyniki pracy magisterskiej oraz wnioski z niej
wyciągnięte, a także możliwości na jej dalszy rozwój przygotowanego
rozwiązania.

\section{Wyniki pracy}

W ramach tej pracy magisterskiej technologia \emph{Sirius Web} została
wykorzystana do stworzenia edytora diagramów opisujących aplikacje obliczeniowe
systemu \emph{BalticLSC}. Wykorzystuje on przygotowany specjalnie w tym celu
metamodel \gls{EMF} języka \gls{CAL}. Do edytora zostały dodane narzędzia i
funkcjonalności pomagające w efektywniejszym i szybszym tworzeniu modeli.
Ocenione zostały także możliwości edytora i porównane z możliwościami
konkurencyjnej poprzedniej wersji edytora bazującego na metamodelach \gls{EMF},
czyli \emph{Sirius Desktop}.

Edytor użyty jest domyślnie jako część interfejsu przykładowej aplikacji
wykorzystującej bibliotekę \emph{React}.
Jako jedna z części tej pracy udostępniono interfejs programistyczny
pozwalający na wyświetlenie edytora w dowolnej aplikacji przeglądarkowej
wykorzystując wyłącznie funkcje w języku \emph{JavaScript}, bez wymogu używania
konkretnej biblioteki.
Pozwala to na wykorzystanie go jako
alternatywny edytor diagramów platformy \emph{BalticLSC}, która wykorzystuje
bibliotekę \emph{Vue.js}. Zostało to
zademonstrowane poprzez przygotowanie aplikacji przeglądarkowej wyświetlającej
edytor diagramów jako część swojego interfejsu.

Edytor domyślnie pozwala na wyświetlenie modeli w formie diagramów zgodnie z
definicją reprezentacji z metamodelu \gls{EMF} oraz na podstawowe możliwości
edycji tych modeli. Zachowana jest funkcjonalność interpretacji języka
\gls{AQL} do wyrażenia bardziej zaawansowanej logiki czy wyrażeń opisujących
właściwości metamodelu. Jednak nie wszystkie mozliwości metamodeli \gls{EMF}
wspierane przez \emph{Sirius Desktop} są obsługiwane przez \emph{Sirius Web}.
Niektóre elementy są również inaczej wyświetlane co sprawia, że metamodel
należy szczególnie dostosować pod \emph{Sirius Web}.

Różnice między \emph{Sirius Web} i \emph{Sirius Desktop}, a także błędy w
funkcjonowaniu oraz brakujące funkcjonalności edytora sprawiły, że autorom
projektu zostało zgłoszone 20 usterek w repozytorium projektu
\texttt{sirius-components} na platformie \emph{GitHub}. Dwie z nich zostały
rozwiązane podczas pisania pracy magisterskiej. Część problemów posiadała
alternatywne rozwiązania, które pomimo swoich wad pozwalały uzyskać zamierzony
efekt. W innych przypadkach brakujące funkcjonalności należało zaimplementować
samemu.

Funkcjonalnościami dodanymi do edytora w ramach pracy magisterskiej są
mechanizm walidacji semantycznej modeli języka \gls{CAL} oraz implementacja
przybornika wyświetlającego listę dostępnych modułów aplikacyjnych z platformy
\emph{BalticLSC}. Modyfikacja kodu serwera aplikacyjnego zazwyczaj była
łatwiejsza od modyfikacji kodu aplikacji przeglądarkowej, ponieważ dzięki
mechanizmowi wstrzykiwania zależności z platformy \emph{Java Spring}
często wystarczyło jedynie dodać nowe klasy i ewentualnie zmienić kilka linii w
już istniejących.

Zmiana kodu interfejsu użytkownika aplikacji przeglądarkowej była pracochłonna
i wiązała się ze skopiowaniem kodu źródłowego tych komponentów, które mają być
modyfikowane. \emph{Sirius Web} nie udostępnia prostych metod na dodanie
własnych elementów interfejsu do istniejących komponentów, więc kopiowanie i
edycja ich kodu źródłowego jest jedynym rozwiązaniem. Dodaje to natomiast pracy
przy późniejszych aktualizacjach do nowej wersji ponieważ należy upewnić się
czy kod skopiowanych komponentów został zmieniony. Jeżeli tak, należy odtworzyć
te zmiany, ponieważ w przeciwnym wypadku aplikacja może działać niepoprawnie.

Wykorzystanie, a później modyfikacja edytora \emph{Sirius Web} była utrudniona
przez brak dokumentacji, niewielką liczbę publicznie dostępnych przykładów
wykorzystania rozwiązania oraz małą liczbę użytkowników zadających pytania,
których odpowiedzi byłyby widoczne publicznie. Te wszystkie aspekty wynikają z
krótkiego czasu życia projektu, ponieważ został on rozpoczęty w 2018 roku, a
dopiero w 2020 roku jego kod źródłowy został udostępniony na platformie
\emph{GitHub}. Praca nad edytorem wymagała poświęcenia sporej ilości czasu na
zrozumienie jak \emph{Sirius Web} działa poprzez czytanie kodu źródłowego
przykładowej aplikacji z repozytorium \texttt{sirius-web}, jak i poszczególnych
klas biblioteki z repozytorium \texttt{sirius-components}. Gdy znalezienie
odpowiedzi na nurtujące pytania zajmowało zbyt dużo czasu, zadawano pytanie
autorom projektu za pomocą \emph{GitHub Issues}.

Umożliwienie edycji modeli w przeglądarce ma sporo zalet w porównaniu do
edytorów funkcjonujących jako aplikacje natywne, takie jak \emph{Sirius
	Desktop}. Nie wymaga on procesu instalacji, można szybciej udostępnić
model innym użytkownikom, łatwiej jest przeprowadzać aktualizację metamodelu do
nowszej wersji, zmiany w modelach są automatycznie propagowane do wszystkich
użytkowników w czasie rzeczywistym. Sam interfejs został też odświeżony i
wygląda bardziej estetycznie w porównaniu do \emph{Sirius Desktop}.

Wykorzystanie sieci Internet do obsługi edytora ma też swoje wady. Niektóre z
nich są szczególnie problematyczne i widoczne z uwagi na przyjęty schemat
komunikacji między aplikacją przeglądarkową a serwerem aplikacyjnym. Na wolnych
lub niestabilnych łączach edytor może działać Wykorzystanie sieci Internet do
obsługi edytora ma też swoje wady. Niektóre z nich są szczególnie
problematyczne i widoczne z uwagi na przyjęty schemat komunikacji między
aplikacją przeglądarkową a serwerem aplikacyjnym. Na wolnych lub niestabilnych
łączach edytor może działać Wykorzystanie sieci Internet do obsługi edytora ma
też swoje wady. Niektóre z nich są szczególnie problematyczne i widoczne z
uwagi na przyjęty schemat komunikacji między aplikacją przeglądarkową a
serwerem aplikacyjnym. Na wolnych lub niestabilnych łączach aplikacja edytor
może działać ze sporym opóźnieniem i powodować frustrację podczas jego obsługi,
bo każda zmiana modelu musi zostać wprowadzona przez serwer aplikacyjny.
Problem pogarsza fakt, że przy każdej zmianie wysyłane są wszystkie informacje
o modelu zamiast jedynie informacji o zmienionych fragmentach.

Projekt \emph{Sirius Web} jest aktywnie rozwijany. W ciągu miesiąca (od 15
grudnia 2021 r.\ do 15 stycznia 2022 r.) zostało wydanych 9 nowych wersji tego
oprogramowania. Niektóre z nich wymagały jedynie pobrania nowych plików
źródłowych, a inne z nich wprowadzały zmiany łamiące kompatybilność wsteczną i
wymagające niekiedy znacznych zmian w kodzie źródłowym. Zmiany te były najpierw
wprowadzane przez zespół pracujący nad tą technologią w repozytorium
\texttt{sirius-web}, więc można było je wykorzystać podczas aktualizacji
własnej kopii edytora. Było to natomiast tym trudniejsze, im bardziej edytor
został przez nas zmodyfikowany, ponieważ wtedy kod bardziej się róznił i
narzędzia umożliwiające automatyczne wprowadzanie zmian (pochodzące z systemu
kontroli wersji \emph{git}) nie wiedziały jak je wprowadzić.

Podsumowując, wszystkie zamierzone cele pracy magisterskiej zostały osiągnięte.

\section{Wnioski}

Sirius Web jest w fazie rozwoju, ale można już go użyć. Brakuje dokumentacji,
więc wymagany jest \textit{reverse engineering}, czytanie kodu źródłowego,
metoda prób i błędów, debuggowanie aplikacji, zgłaszanie usterek lub zadawanie
pytań w repozytorium projektu.

Występują różnice między Sirius Web a Sirius Desktop.

Najpierw sprawdzić czy funkcjonalność działa, a dopiero potem spędzać czas na
dodanie jej w metamodelu

\section{Możliwości na rozwój}

Wymienić co można zrobić dalej:

\begin{itemize}
	\item integracja z BalticLSC
	\item dodanie większej liczby reguł walidacji semantycznej
\end{itemize}
