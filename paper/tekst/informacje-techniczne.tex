\chapter{Informacje techniczne}

\section{Wykorzystane biblioteki}

Tabela z wykorzystywanymi bibliotekami i ich licencjami.

\section{Projekt systemu}

Diagram z backendami oraz frontendami BalticLSC, a także bazą danych
PostgreSQL\@.

\section{Projekt modułów}

Informacje o modułach backendu (projekty Javy w Sirius Web).

\section{Instrukcja wdrożenia}

\subsection{Wymagania}

Docker, lub Java 11, Maven, PostgreSQL

\subsection{Instrukcja instalacji}

Opis: albo Docker, albo instalacja manualna (potwórzenie instrukcji z
\texttt{README.md}).

\subsection{Instrukcja uruchomienia}

Opis: albo Docker, albo manualnie uruchomienie serwera

\section{Instrukcja użycia}

Jak użyć aplikacji do stworzenia prostego modelu

Powinno także zawierać informacje o logowaniu się do BalticLSC\@.

\section{Instrukcja utrzymania}

\subsection{Kopia zapasowa bazy danych}

Jak stworzyć kopię bazy w PostgreSQL\@?

\subsection{Aktualizowanie wersji Sirius Web}

Opis metody aplikowania najnowszych zmian z repozytorium Sirius Web
(generowanie \texttt{git patch} i aplikowanie ich).

\section{Zabezpieczenia}

Brak zabezpieczeń.

\section{Testy akceptacyjne rozwiązania}

Opis kilku testów, które użytkownik może chcieć wykonać aby sprawdzić, czy
rozwiązanie działa poprawnie.

\subsection{Wynik testów akceptacyjnych}

Czy się udały?
