\chapter{Metamodel dla języka CAL}

\section{Język opisu obliczeń w BalticLSC}

\section{Stworzony metamodel EMF dla języka CAL}

Omówienie przygotowanego modelu. Zrzut ekranu przedstawiający model. Omówienie
elementów oraz jakie one mają przełożenie na wykonanie obliczeń przez
BalticLSC\@.

\subsection{Warunkowa zmiana stylu elementów}

Omówienie reguł warunkowej zmiany stylu elementów diagramu:

\begin{enumerate}
	\item \texttt{DataPin} zmieniają ikonę na podstawie swoich \textit{data
		      multiplicity} oraz \textit{token multiplicity}.
	\item \texttt{ComputedDataPin} zmieniają kolor na podstawie swojego
	      \textit{data binding}.
\end{enumerate}

\subsection{Narzędzia edytora diagramów}

Omówienie dodanych \textit{Tools} z Sirius:

\begin{itemize}
	\item usunięcie \texttt{UnitCall} lub \texttt{ApplicationDataPin} usuwa również powiązane \texttt{DataFlow}
	\item usunięcie \texttt{ComputedDataPin} z poziomu edytora diagramów nie jest możliwe
	\item ograniczenia na tworzenie \texttt{DataFlow} tak, aby były semantycznie poprawne
	\item automatyczne usuwanie i tworzenie \texttt{ComputedDataPin} po zmianie \texttt{ComputationUnitRelease} dla danego \texttt{UnitCall}
	\item okno dialogowe ułatwiające tworzenie \texttt{UnitCall} dla istniejącego \texttt{ComputationUnitRelease}
	\item \ldots
\end{itemize}

\subsection{Reguły walidacyjne powiązane z
	metamodelem}\label{sec:regulky-walidacyjne-metamodel}

Omówienie \textit{semantic validation rule} z pliku \texttt{*.odesign}, które
działają w Sirius Desktop.

\subsection{Testy metamodelu}

Omówienie dodanych testów jednostkowych modelu (głównie dotyczy automatycznego
zarządzania \texttt{ComputedDataPin} w zależnosci od
\texttt{ComputationUnitRelease} dla konkretnego \texttt{UnitCall}).
