\chapter{Dostosowanie Sirius Web dla systemu BalticLSC}

\section{Użycie metamodelu języka CAL w Sirius Web}

Opis jak wykorzystać metamodel EMF w Sirius Web.

\begin{itemize}
	\item konfiguracja Maven
	\item odpowiednia modyfikacja klas Javy w Sirius Web
\end{itemize}

\noindent oraz jaki rezultat uzyskano.

\section{Integracja przybornika BalticLSC w Sirius Web}

Opis problemu dodawania nowych \texttt{UnitCall} do diagramu --- trzeba
zdefiniować \texttt{ComputationUnitRelease} w diagramie.

Rozwiązanie (zaczerpnięte z edytora diagramów BalticLSC) --- przybornik
(toolbox).

Opis jak to zrobiono, od strony backendu jak i frontendu. Omówienie trudności w
modyfikacji interfejsu użytkownika Sirius Web (trzeba było skopiować kod
źródłowy niektórych komponentów z biblioteki \textit{Sirius Components} do kodu
aplikacji Sirius Web, ponieważ komponenty te nie umożliwiały modyfikacji
interfejsu i wstawiania do nich nowych elementów --- najlepiej dać zrzut ekranu
co można było łatwo zmienić, a co wymagało skopiowania kodu).

\section{Walidacja semantyczna modelu}

Informacja o informacjach diagnostycznych udostępnianych domyślnie przez Sirius
Web.

Brak uruchamiania reguł semantycznych zdefiniowanych w
metamodelu~\ref{sec:reguly-walidacyjne-metamodel}.

Opis dodanego rozwiązania (własne klasy Javowe które zwracają listę informacji
diagnostycznych, oraz strumieniowanie ich do przeglądarki wykorzystując
istniejące rozwiązanie do walidacji).

\section{Użycie edytora Sirius Web w BalticLSC}

Omówienie przygotowanego planu integracji.
