Z ramach tej pracy magisterskiej został przygotowany graficzny edytor modeli
opisujących obliczenia rozproszone w systemie \BalticLSC{} bazując na
eksperymentalnej technologii \SiriusWeb{}. Została ona oceniona i porównana ze
znanym w tej kategorii rozwiązaniem \SiriusDesktop{}.

Programy komputerowe używają modeli do reprezentacji świata rzeczywistego.
W~ten~sposób mogą odwzorować rzeczywistość i symulować jej modyfikacje.
Struktura
modeli, a~więc~występujące w nich obiekty i ich właściwości, może również być
opisana
modelem. Jest~on~na~wyższym poziomie abstrakcji niż model, który opisuje, i
nosi nazwę \emph{metamodelu}.
Na jego bazie można wygenerować edytor graficzny umożliwiający przeglądanie
i edycję modeli opisywanych przez ten metamodel. \emph{\acrlong{EMF}} jest
technologią pozwalającą na przygotowanie edytora graficznego modeli
\SiriusDesktop{},
będącego rozszerzeniem natywnej aplikacji \Eclipse{}. Nowym rozwiązaniem w tej
kategorii jest \SiriusWeb{} umożliwiający udostępnienie graficznego edytora
modeli w przeglądarce internetowej.

Przygotowane rozwiązanie pozwala stworzyć model opisujący schemat obliczeń,
które będą wykonane w systemie \BalticLSC{}. Jest to platforma do obliczeń
rozproszonych, rozwijana między innymi przez Politechnikę Warszawską,
umożliwiająca przeprowadzenie obliczeń na~dużą skalę używając potencjalnie
wielu węzłach obliczeniowych jednocześnie.
Przepływ danych przez różne dostępne w systemie moduły obliczeniowe jest
reprezentowany na diagramie.

\SiriusDesktop{} i \SiriusWeb{}, pomimo bazowania na tych samych metamodelach,
oferują różne możliwości i funkcjonalności. Są również inaczej zbudowane i mają
inne podejście do~przechowywania informacji o modelach. Niektóre elementy
edytora \SiriusDesktop{}, na~przykład możliwość przeprowadzenia walidacji
semantycznej modelu, są niedostępne w~\SiriusWeb{} i zostały zaimplementowane w
tej pracy magisterskiej. W celu sprawdzenia rozszerzalności edytora dodano do
niego elementy interfejsu graficznego pozwalające wygodnie wykorzystać
dostępne moduły obliczeniowe na diagramie. Opisano również jak można
zintegrować przygotowany edytor z aplikacją przeglądarkową \BalticLSC{}.

Technologia \SiriusWeb{} zawiera również wady. Znalezionych 20 usterek zostało
zgłoszonych autorom projektu, a wrażenia z wykorzystania tego rozwiązania
zostały opisane w dalszej części tekstu pracy magisterskiej.
