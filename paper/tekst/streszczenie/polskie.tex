W ramach tej pracy magisterskiej został przygotowany graficzny edytor modeli
opisujących obliczenia rozproszone w systemie \BalticLSC{} bazując na
eksperymentalnej technologii \SiriusWeb{}. Została ona oceniona i porównana ze
znanym w tej kategorii rozwiązaniem \SiriusDesktop{}. Do bazowego edytora
\SiriusWeb{} zostały też dodane brakujące mechanizmy walidacji edytowanych
modeli.

Dla języka \emph{\acrfull{CAL}} pozwalającego reprezentować aplikacje
obliczeniowe w systemie \BalticLSC{} został przygotowany metamodel w formacie
\Ecore{} używanym w technologii \emph{\acrfull{EMF}}. Został on następnie użyty
do~przygotowania graficznego edytora modeli za pomocą \SiriusWeb{}. Wczesna
faza rozwoju tej technologii i brak jej dokumentacji technicznej spowodował, że
wymagało to~wykorzystania wiedzy z wielu różnych źródeł, a także
metody prób i błędów, aby dostosować pakiety języka \Java{} wygenerowane za
pomocą \emph{\acrshort{EMF}} do formatu wymaganego przez \SiriusWeb{}.

Prezentowany metamodel języka \emph{\acrshort{CAL}} zawiera w sobie reguły
walidacji semantycznej wyrażone jako \emph{Semantic Validation Rule} formatu
\Ecore{}. Nie są jednak one sprawdzane przez \SiriusWeb{}, który wyświetla
użytkownikowi jedynie informacje diagnostyczne z walidacji składniowej modelu.
Do edytora dodano mechanizm walidacji semantycznej poprzez modyfikację kodu
serwera aplikacyjnego i uruchamianie w nim odpowiednich reguł napisanych w
języku \Java{}. Inną funkcjonalnością, o którą wzbogacono edytor, jest
przybornik pozwalający na łatwe wykorzystanie dostępnych jednostek
obliczeniowych pobieranych z platformy \BalticLSC{}.

Przygotowany edytor modeli można wykorzystać jako część dowolnej innej
aplikacji przeglądarkowej, co zademonstrowano tworząc taką przykładową
aplikację. W pracy opisany jest też proponowany plan integracji edytora z
platformą \BalticLSC{}.

Praca zawiera również ocenę technologii \SiriusWeb{} i porównanie jej z
poprzednikiem --- \SiriusDesktop{}. Zademonstrowano różnice w wyświetlaniu
elementów diagramu, możliwości modyfikacji edytora i dodawania w nim nowych
funkcjonalności, kompletności obsługi formatu \Ecore{}, łatwości w
wykorzystaniu oraz utrzymaniu. Podczas pracy znaleziono 20 usterek
w~technologii \SiriusWeb{}, które zostały zgłoszone autorom projektu.
