{
\selectlanguage{english}
% NOTE: Interlinia ustawiona w EE-dyplom.cls nie jest stosowana dla języka
% angielskiego. Należy ją ustawić manualnie.
\setstretch{1.2213}
In this thesis a graphical model editor for expressing distributed
computations in~the~\BalticLSC{} system is presented. The editor is based on
the experimental \SiriusWeb{} technology. It~is~assessed and
compared with the more mature \SiriusDesktop{} as part of this thesis.

A metamodel in the \Ecore{} format used in the \emph{\acrfull{EMF}}
technology was prepared for the \emph{\acrfull{CAL}}. It allows
representing computation applications in the \BalticLSC{} system. It was
used to prepare a~graphical model editor using the \SiriusWeb{} technology. The
early stage of
this project combined with the lack of its technical documentation meant
that attempting to~use~it~proved to be a challenge and required combining
knowledge from multiple sources and the trial and~error approach. That is
because of the steps required to adjust the
\Java{} packages generated by \emph{\acrshort{EMF}} to the format that
\SiriusWeb{} expects, which were not clearly described.

The presented \emph{\acrshort{CAL}} metamodel contains \emph{Semantic
	Validation Rules} described in the \Ecore{} format, which help detect
invalid
models. However, they are not used in \SiriusWeb{}. It only displays diagnostic
information from the syntactic model validation. A mechanism for semantic model
validation was added to the editor by modifying the application server's code.
It relies on invoking rules represented as classes in the \Java{} programming
language. Another major functionality added to the editor is a toolbox which
allows convenient use of the available computation units fetched from the
\BalticLSC{} platform.

The presented model editor can be used as a part of another web application,
which was~demonstrated by preparing such an example application. The thesis
also proposes a~plan of~integrating the editor with the \BalticLSC{}
platform.

The thesis includes an assessment of the \SiriusWeb{} technology and a
comparison with its~predecessor --- \SiriusDesktop{}. Demonstrated are the
differences in how elements are~displayed, the ease of using, modifying,
maintaining, and enhancing the editor, as well as~the~\Ecore{} format support.
20 issues of \SiriusWeb{} were found in the process of preparing this
thesis. All of them were reported to the project's maintainers.
}
