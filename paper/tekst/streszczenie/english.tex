{
\selectlanguage{english}
In this thesis a graphical model editor for expressing distributed
computations in~the~\BalticLSC{} system is presented. The editor is based on
the experimental \SiriusWeb{} technology. It~is~assessed and
compared with the more mature \SiriusDesktop{} as part of this thesis.

Computer programs use models to represent the external world. In this way they
can imitate the reality and simulate modifying it. The objects and their
attributes, which constitute the~structure of the model, can also be described
by a separate model. It has a higher abstraction level and is called a
\emph{metamodel}.
Graphical editors can be based on~it~and~allow viewing and~editing models
described by that metamodel. \emph{\acrlong{EMF}} is~a~technology used to
create a graphical model editor \SiriusDesktop{}. It is an extension
of~the~\Eclipse{} native application. A new solution in this area is
\SiriusWeb{}. It
allows creating a~graphical model editor running in a web browser.

The solution presented in this thesis allows creating models which describe
the~order and~types of computations run in the \BalticLSC{} system. It is a
platform for distributed computing developed, among others, by the Warsaw
University of
Technology. It allows executing large--scale % chktex 8
computations, potentially on multiple worker nodes in parallel.
Diagrams describe how~the~data flows through various available computation
units.

\SiriusDesktop{} and \SiriusWeb{}, despite being based on the same metamodels,
differ in~the~capabilites and functionalities they offer. Their architecture
and model persistence format are~also different. Some actions available in
\SiriusDesktop{}, for example invoking semantic model validation, are not
present in \SiriusWeb{} and had to be implemented as~part of~this thesis.
To assess the editor's extensibility a new user interface feature was added.
It~allows conveniently using one of the available computation modules in the
diagram. The~steps required to embed the prepared solution into the
\BalticLSC{} web application were also described in this thesis.

The \SiriusWeb{} technology has its downsides. The 20 issues found within this
thesis were reported to the project's maintainers. The technology was also
assessed in thesis text.
}
