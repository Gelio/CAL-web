% ---------------------------------------------
% Szablon prac dyplomowych na Wydziale Elektrycznym PW,
% zgodny z Zarządzeniami JM Rektora PW:
% 24/2016, 43/2016, 57/2016.
% Większość opcji ustawiona zgodnie z zaleceniami.
% Główne różnice - dla poprawy czytelności pracy:
%   * stopień pisma 12 pt (zamiast 11 pt)
%   * zwiększone światło międzywierszowe
%   * wyróżniające się nagłówki rozdziałów
%   * wyróżnione kolorem, klikalne odnośniki
%
% Jak używać szablonu:
% 1) Szablon jest przygotowany dla XeLaTeX więc jeśli używasz
% Overleaf to otwórz "Menu" i zmień "Compiler" na "XeLaTeX".
% 2) ustaw niżej właściwy typ pracy w linii "documentclass"
% wybierając dla "thesis" jedną opcję z: inz, mgr, bsc, msc
% 3) spersonalizuj pola w pliku "config.tex"
% 4) modyfikuj, zmieniaj i dodawaj treść w katalogu "tekst"
% 5) dodawaj rysunki w katalogu "rysunki" (gfx raczej nie używaj)
%
% Dziękuję innym Autorom, których szablonami mogłem się inspirować:
% prof. dr hab. inż. Jacek Starzyński, Wydział Elektryczny, PW
% Artur M. Brodzki i Piotr Woźniak, Wydział EiTI, PW
%
% Licencja szablonu: CC-BY 4.0
% https://creativecommons.org/licenses/by/4.0/
% Autor: Łukasz Makowski <lukasz.makowski@ee.pw.edu.pl>
% https://github.com/SP5LMA/EE-dyplom
% ---------------------------------------------

% Wybierz rodzaj pracy dyplomowej
% Pick thesis type
\documentclass[thesis=bsc,faculty=ee]{EE-dyplom} % thesis=[inz|mgr|bsc|msc]
% Konfiguracja - do personalizacji
% Configuration - to be personalized
\instytut{Zakład Elektrotechniki Teoretycznej i Informatyki Stosowanej}
\kierunek{Informatyka Stosowana}
\specjalnosc{Inżynieria Oprogramowania}
\title{Mechanizm walidacji modeli dla systemu BalticLSC w środowisku Sirius
	Web}
\engtitle{Validation mechanism for BalticLSC models in the Sirius Web
	environment}
\album{276887}
\author{inż.\ Grzegorz Rozdzialik}
\promotor{dr hab.\ inż.\ Michał Śmiałek}
\date{2022}

\streszczeniepracy{
	W ramach tej pracy magisterskiej został przygotowany graficzny edytor modeli
opisujących obliczenia rozproszone w systemie \BalticLSC{} bazując na
eksperymentalnej technologii \SiriusWeb{}. Została ona oceniona i porównana ze
znanym w tej kategorii rozwiązaniem \SiriusDesktop{}.

Programy komputerowe używają modeli do reprezentacji świata rzeczywistego.
W~ten~sposób mogą odwzorować rzeczywistość i symulować jej modyfikacje.
Struktura
modeli, a~więc~występujące w nich obiekty i ich właściwości, może również być
opisana
modelem. Jest~on~na~wyższym poziomie abstrakcji niż model, który opisuje, i
nosi nazwę \emph{metamodelu}.
Na jego bazie można wygenerować edytor graficzny umożliwiający przeglądanie
i edycję modeli opisywanych przez ten metamodel. \emph{\acrlong{EMF}} jest
technologią pozwalającą na przygotowanie edytora graficznego modeli
\SiriusDesktop{},
będącego rozszerzeniem natywnej aplikacji \Eclipse{}. Nowym rozwiązaniem w tej
kategorii jest \SiriusWeb{} umożliwiający udostępnienie graficznego edytora
modeli w przeglądarce internetowej.

Przygotowane rozwiązanie pozwala stworzyć model opisujący schemat obliczeń,
które będą wykonane w systemie \BalticLSC{}. Jest to platforma do obliczeń
rozproszonych, rozwijana między innymi przez Politechnikę Warszawską,
umożliwiająca przeprowadzenie obliczeń na~dużą skalę używając potencjalnie
wielu węzłów obliczeniowych jednocześnie.
Przepływ danych przez różne dostępne w systemie moduły obliczeniowe jest
reprezentowany na diagramie.

\SiriusDesktop{} i \SiriusWeb{}, pomimo bazowania na tych samych metamodelach,
oferują różne możliwości i funkcjonalności. Są również inaczej zbudowane i mają
inne podejście do~przechowywania informacji o modelach. Niektóre elementy
edytora \SiriusDesktop{}, na~przykład możliwość przeprowadzenia walidacji
semantycznej modelu, są niedostępne w~\SiriusWeb{} i zostały zaimplementowane w
tej pracy magisterskiej. W celu sprawdzenia rozszerzalności edytora dodano do
niego elementy interfejsu graficznego pozwalające wygodnie wykorzystać
dostępne moduły obliczeniowe na diagramie. Opisano również jak można
zintegrować przygotowany edytor z aplikacją przeglądarkową \BalticLSC{}.

Technologia \SiriusWeb{} zawiera również wady. Znalezionych 20 usterek zostało
zgłoszonych autorom projektu, a wrażenia z wykorzystania tego rozwiązania
zostały opisane w dalszej części tekstu pracy magisterskiej.

}
\slowakluczowe{metamodelowanie, metamodel, EMF, Eclipse Modeling Framework,
	Sirius Desktop, Sirius Web, aplikacja przeglądarkowa, edytor graficzny,
	walidacja modeli, BalticLSC}

\thesisabstract{
	{
\selectlanguage{english}
% NOTE: Interlinia ustawiona w EE-dyplom.cls nie jest stosowana dla języka
% angielskiego. Należy ją ustawić manualnie.
\setstretch{1.2213}
In this thesis a graphical model editor for expressing distributed
computations in~the~\BalticLSC{} system is presented. The editor is based on
the experimental \SiriusWeb{} technology. It~is~assessed and
compared with the more mature \SiriusDesktop{} as part of this thesis.

Computer programs use models to represent the external world. In this way they
can imitate the reality and simulate modifying it. The objects and their
attributes, which constitute the~structure of the model, can also be described
by a separate model. It has a higher abstraction level and is called a
\emph{metamodel}.
Graphical editors can be based on~it~and~allow viewing and~editing models
described by that metamodel. \emph{\acrlong{EMF}} is~a~technology used to
create graphical model editors with \SiriusDesktop{}. It is an extension
of~the~\Eclipse{} native application. A new solution in this area is
\SiriusWeb{}. It
allows creating graphical model editors running in a web browser.

The solution presented in this thesis allows creating models which describe
the~order and~types of computations run in the \BalticLSC{} system. It is a
platform for distributed computing developed, among others, by the Warsaw
University of
Technology. It allows executing large--scale % chktex 8
computations, potentially on multiple worker nodes in parallel.
Diagrams describe how~the~data flows through various available computation
units.

\SiriusDesktop{} and \SiriusWeb{}, despite being based on the same metamodels,
differ in~the~capabilities and functionalities they offer. Their architecture
and model persistence format are~also different. Some actions available in
\SiriusDesktop{}, for example invoking semantic model validation, are not
present in \SiriusWeb{} and had to be implemented as~part of~this thesis.
To assess the editor's extensibility, a new user interface feature was added.
It~allows conveniently using the available computation modules in the
diagram. The~steps required to embed the prepared solution into the
\BalticLSC{} web application were also described in this thesis.

The \SiriusWeb{} technology has its downsides. The 20 issues found within this
thesis were reported to the project's maintainers. The technology was also
assessed in the thesis text.
}

}
\thesiskeywords{metamodeling, metamodel, EMF, Eclipse Modeling Framework,
	Sirius Desktop, Sirius Web, web application, graphical editor,
	model validation, BalticLSC}


% Tu zaczyna się dokument
% Here is the beginning of the document
\begin{document}
% Strony nagłówkowe
% Headers
\frontpages

% Właściwa treść jest w pliku tekst/main.tex
% Real contents is in tekst/main.tex
\chapter{Wstęp}

Wspomnieć czym jest BalticLSC.

\section{Cel pracy}

Cel --- weryfikacja jakich możliwości daje wykorzystanie Sirius Web w zakresie
stworzenia edytora diagramów reprezentujących modele BalticLSC, a następnie
dodanie mechanizmów pozwalających na sprawdzenie poprawności semantycznej
modelu.

\section{Istniejące rozwiązania}

Opis alternatywnych rozwiązań --- budowanie rozwiązań specyficznych dla
konkretnej domeny używając ogólnych bibliotek warstwy UI do diagramów, oraz
dodawanie do nich samemu walidacji. Również konieczność tworzenia serwera
obsługującego te diagramy.

\section{Motywacja}

Dzięki Sirius Web oraz metamodelowaniu można zbudować edytor diagramów wraz z
walidacją semantyczną dla dowolnej domeny.

\section{Zakres pracy}

Co zostało zrobione w ramach pracy:

\begin{enumerate}
	\item Stworzenie metamodelu języka CAL w EMF\@.
	\item Wykorzystanie tego metamodelu w Sirius Web.
	\item Porównanie możliwości Sirius Web i Sirius Desktop. Zgłoszenie usterk autorom Sirius Web poprzez GitHub.
	\item Dodanie do modelu elementów usprawniających pracę z nim (automatyzacja niektórych czynności, dodanie ograniczeń utrudniających zrobienie błędu).
	\item Modyfikacja interfejsu użytkownika Sirius Web poprzez dodanie do niego przybornika z BalticLSC\@. Przybornik umożliwia w łatwy sposób dodanie nowych \texttt{UnitCall} do modelu.
	\item Dodanie mechanizmu walidacji semantycznej modelu sprawdzającego poprawność modelu ze zdefinionwanymi w języku Java regułami.
	\item Stworzenie planu integracji rozwiązania z BalticLSC\@.
\end{enumerate}

\noindent Co zostaje poza zakresem pracy:

\begin{enumerate}
	\item Integracja rozwiązania jako alternatywnego edytora diagramów dla systemu Sirius Web.
\end{enumerate}

\chapter{Tworzenie edytorów graficznych na bazie metamodeli}

\section{Metamodelowanie}

\section{Edytory graficzne na podstawie metamodeli}

\chapter{Metamodel dla języka CAL}

W tym rozdziale zostanie omówione przygotowane rozwiązanie.

\section{Język opisu obliczeń w BalticLSC}

\section{Stworzony metamodel EMF dla języka CAL}

Omówienie przygotowanego modelu. Zrzut ekranu przedstawiający model. Omówienie
elementów oraz jakie one mają przełożenie na wykonanie obliczeń przez
BalticLSC\@.

\subsection{Warunkowa zmiana stylu elementów}

Omówienie reguł warunkowej zmiany stylu elementów diagramu:

\begin{enumerate}
	\item \texttt{DataPin} zmieniają ikonę na podstawie swoich \textit{data
		      multiplicity} oraz \textit{token multiplicity}.
	\item \texttt{ComputedDataPin} zmieniają kolor na podstawie swojego
	      \textit{data binding}.
\end{enumerate}

\subsection{Narzędzia edytora diagramów}

Omówienie dodanych \textit{Tools} z Sirius:

\begin{itemize}
	\item usunięcie \texttt{UnitCall} lub \texttt{ApplicationDataPin} usuwa również powiązane \texttt{DataFlow}
	\item usunięcie \texttt{ComputedDataPin} z poziomu edytora diagramów nie jest możliwe
	\item ograniczenia na tworzenie \texttt{DataFlow} tak, aby były semantycznie poprawne
	\item automatyczne usuwanie i tworzenie \texttt{ComputedDataPin} po zmianie \texttt{ComputationUnitRelease} dla danego \texttt{UnitCall}
	\item okno dialogowe ułatwiające tworzenie \texttt{UnitCall} dla istniejącego \texttt{ComputationUnitRelease}
	\item \ldots
\end{itemize}

\subsection{Reguły walidacyjne powiązane z
	metamodelem}\label{sec:regulky-walidacyjne-metamodel}

Omówienie \textit{semantic validation rule} z pliku \texttt{*.odesign}, które
działają w Sirius Desktop.

\subsection{Testy metamodelu}

Omówienie dodanych testów jednostkowych modelu (głównie dotyczy automatycznego
zarządzania \texttt{ComputedDataPin} w zależnosci od
\texttt{ComputationUnitRelease} dla konkretnego \texttt{UnitCall}).

\chapter{Dostosowanie Sirius Web dla systemu BalticLSC}

\section{Użycie metamodelu języka CAL w Sirius Web}

Opis jak wykorzystać metamodel EMF w Sirius Web.

\begin{itemize}
	\item konfiguracja Maven
	\item odpowiednia modyfikacja klas Javy w Sirius Web
\end{itemize}

\noindent oraz jaki rezultat uzyskano.

\section{Integracja przybornika BalticLSC w Sirius Web}

Opis problemu dodawania nowych \texttt{UnitCall} do diagramu --- trzeba
zdefiniować \texttt{ComputationUnitRelease} w diagramie.

Rozwiązanie (zaczerpnięte z edytora diagramów BalticLSC) --- przybornik
(toolbox).

Opis jak to zrobiono, od strony backendu jak i frontendu. Omówienie trudności w
modyfikacji interfejsu użytkownika Sirius Web (trzeba było skopiować kod
źródłowy niektórych komponentów z biblioteki \textit{Sirius Components} do kodu
aplikacji Sirius Web, ponieważ komponenty te nie umożliwiały modyfikacji
interfejsu i wstawiania do nich nowych elementów --- najlepiej dać zrzut ekranu
co można było łatwo zmienić, a co wymagało skopiowania kodu).

\section{Walidacja semantyczna modelu}

Informacja o informacjach diagnostycznych udostępnianych domyślnie przez Sirius
Web.

Brak uruchamiania reguł semantycznych zdefiniowanych w
metamodelu~\ref{sec:regulky-walidacyjne-metamodel}.

Opis dodanego rozwiązania (własne klasy Javowe które zwracają listę informacji
diagnostycznych, oraz strumieniowanie ich do przeglądarki wykorzystując
istniejące rozwiązanie do walidacji).

\section{Użycie edytora Sirius Web w BalticLSC}

Omówienie przygotowanego planu integracji.

\chapter{Ocena Sirius Web}

\section{Różnice między Sirius Web a Sirius Desktop}

Omówienie różnic oraz usterek zgłoszonych przeze mnie w repozytorium Sirius
Web.

\section{Użycie własnego metamodelu}

\section{Dodawanie funkcjonalności do edytora}

\chapter{Informacje techniczne}

\section{Wykorzystane biblioteki}

Tabela z wykorzystywanymi bibliotekami i ich licencjami.

\section{Projekt systemu}

Diagram z backendami oraz frontendami BalticLSC, a także bazą danych
PostgreSQL\@.

\section{Projekt modułów}

Informacje o modułach backendu (projekty Javy w Sirius Web).

\section{Instrukcja wdrożenia}

\subsection{Wymagania}

Docker, lub Java 11, Maven, PostgreSQL

\subsection{Instrukcja instalacji}

Opis: albo Docker, albo instalacja manualna (potwórzenie instrukcji z
\texttt{README.md}).

\subsection{Instrukcja uruchomienia}

Opis: albo Docker, albo manualnie uruchomienie serwera

\section{Instrukcja użycia}

Jak użyć aplikacji do stworzenia prostego modelu

Powinno także zawierać informacje o logowaniu się do BalticLSC\@.

\section{Instrukcja utrzymania}

\subsection{Kopia zapasowa bazy danych}

Jak stworzyć kopię bazy w PostgreSQL\@?

\subsection{Aktualizowanie wersji Sirius Web}

Opis metody aplikowania najnowszych zmian z repozytorium Sirius Web
(generowanie \texttt{git patch} i aplikowanie ich).

\section{Zabezpieczenia}

Brak zabezpieczeń.

\section{Testy akceptacyjne rozwiązania}

Opis kilku testów, które użytkownik może chcieć wykonać aby sprawdzić, czy
rozwiązanie działa poprawnie.

\subsection{Wynik testów akceptacyjnych}

Czy się udały?

\chapter{Podsumowanie}

\section{Wyniki pracy}

Co zostało osiągnięte? Czy cel został zdobyty?

\section{Wnioski}

Sirius Web jest w fazie rozwoju, ale można już go użyć. Brakuje dokumentacji,
więc wymagany jest \textit{reverse engineering}, czytanie kodu źródłowego,
metoda prób i błędów, debuggowanie aplikacji, zgłaszanie usterek lub zadawanie
pytań w repozytorium projektu.

Występują różnice między Sirius Web a Sirius Desktop.

\section{Możliwości na rozwój}

Wymienić co można zrobić dalej:

\begin{itemize}
	\item integracja z BalticLSC
	\item dodanie większej liczby reguł walidacji semantycznej
\end{itemize}

\section{Sekcje, o których pomyślałem, ale pomijam}\label{sec:pominiete-sekcje}

W tej sekcji wymienię sekcje, o których pomyślałem, że mogą się znaleźć w pracy
magisterskiej. Ostatecznie chciabym je pominąć ze względu na charakter pracy
--- praca jest badawcza, a nie polegała na wytworzeniu nowego oprogramowania.
Pominięte sekcje to:

\begin{enumerate}
	\item wykorzystana metoda wytwarzania oprogramowania
	\item FURPS (wymagania funkcjonalne i niefunkcjonalne)
\end{enumerate}

Ta sekcja (\ref{sec:pominiete-sekcje}) zostanie usunięta z pracy magisterskiej
i służy tylko ułatwieniu komunikacji z promotorem w sprawie ustalenia nagłówków
sekcji pracy magisterskiej.


% Bibliografia - musi być
% Bibliography - must exist
\bibliografia

% Strony końcowe - można zakomentować, jeśli zbędne
% Additional pages - comment out if not needed

% Wykaz symboli i skrótów - patrz opis w tekście przykładowym
\acronymslist
% Spis rysunków
\listoffigures
% Spis tabel
\listoftables
% Załączniki (plik appendices.tex)
\easyappendices
\end{document}
